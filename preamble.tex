%%% Class and paper %%%
\documentclass[a5paper, twoside, 10pt]{book}
\usepackage[top=2.0cm, bottom=2.0cm, left=2.0cm, right=2.0cm]{geometry}
%\usepackage[top=1.0cm, bottom=1.0cm, left=1.0cm, right=1.0cm]{geometry}
%\usepackage[top=2.0cm, bottom=2.0cm, left=2.0cm, right=2.0cm]{gmeometric}

%\usepackage[english]{babel}
%\usepackage{parallel}
\usepackage{setspace}
\setstretch{1.25}

%   \usepackage{ledmac}
%   %%%\footnormal{A}
%   %\foottwocol{A} 
%   \usepackage{ledpar}
%   \maxchunks{3000}
%   \makeatletter
%   \renewcommand{\footnoterule}{\vskip0.5cm\hrule\vskip0.5cm}
%   \let\normalfootnoterule=\hrulefill
%   %%\renewcommand{\footnoteA}[1]{%
%   %%	\refstepcounter{footnoteA}% 
%   %%	\LR{\vfootnoteA{A}{\raggedright\RL{\textsuperscript{\arabic{footnoteA}} #1}}}%
%   %%	\m@mmf@prepare\hbox{\textsuperscript{\arabic{footnoteA}}}}
%   %\renewcommand{\footnoteA}[1]{}
%   \let\OLDthefootnoteA\thefootnoteA
%   \renewcommand{\thefootnoteA}{\OLDthefootnoteA\ }
%   %%\renewcommand*{\bodyfootmarkA}{\hbox{\textsuperscript{(\thefootnoteA)}}}
%   
%   %\renewcommand{\footnoteA}[1]{%
%   %	\refstepcounter{footnoteA}% 
%   %	\LR{%
%   %	\vfootnoteA{A}{\RL{\textsuperscript{\arabic{footnoteA}} #1}}%	
%   %	\m@mmf@prepare}\hbox{\textsuperscript{\arabic{footnoteA}}}}
%   %	\renewcommand{\thefootnoteA}{} 
%   %	\renewcommand*{\bodyfootmarkA}{\hbox{\textsuperscript{(\thefootnoteA)}}}
%   
%   \let\OLDfootnoteA\footnoteA
%   \renewcommand{\footnoteA}[1]{\LR{\OLDfootnoteA{\RL{#1}}}}
%   \newcommand{\footnoteAout}[1]{\RL{\OLDfootnoteA{\RL{#1}}}}
%   \makeatother
%   \firstlinenum{99999}
%   \linenumincrement{99999}


%%% Changemargin %%%
\newenvironment{changemargin}[4]{%
 \begin{list}{}{%
  \setlength{\topsep}{0pt}%
  \setlength{\topmargin}{#1}%
 % \setlength{\bottommargin}{#2}%
  \setlength{\leftmargin}{#3}%
  \setlength{\rightmargin}{#4}%
  \setlength{\listparindent}{\parindent}%
  \setlength{\itemindent}{\parindent}%
  \setlength{\parsep}{\parskip}%
 }%
\item[]}{\end{list}}
\makeatletter

\usepackage{xltxtra}

\newfontfamily\LatinFont[Mapping=tex-text]{SBL Hebrew}
%\newfontfamily\LatinFont[Mapping=tex-text]{Hoefler Text}
\renewcommand{\L}[1]{{\LR{\LatinFont#1}}}
\newcommand{\symbolglyph}[1]{{\fontspec{Symbola}#1}}

\renewcommand{\textsc}[1]{{\fontspec{Palatino Linotype}\scshape#1}}
\newcommand{\textscbf}[1]{\textsc{\textbf{#1}}}
%\setlength\parskip{\medskipamount}
%\setlength\parindent{0pt}

%\newcommand{\hebfootnote}[1]{\R{\footnote{\hspace{-2pt}\HebrewFootnoteFont{}\small#1\medskip}}}
%\let\OLDfootnote\footnote
%\renewcommand{\footnote}[1]{\OLDfootnote{\fontspec{Guttman Hodes Light}#1\medskip}}

\renewcommand{\labelitemi}{\L{~~~•}}
\renewcommand{\labelitemii}{\L{~~~–}}
\renewcommand{\labelenumi}{\arabic{enumi}.\hskip2em}
\renewcommand{\labelenumii}{\alph{enumii}.\hskip2em}



%%% Header/footer %%%
\usepackage{fancyhdr} 
\pagestyle{fancy}
\fancyhead{} % clear all header fields 
\renewcommand{\chaptername}{פרק}
\newcommand{\sectionname}{סעיף}
%\renewcommand{\chaptermark}[1]% 
%{\markboth{\MakeUppercase{#1}}{}} 
%\renewcommand{\sectionmark}[1]% 
%{\markright{\MakeUppercase{#1}}} 
%\renewcommand{\chaptermark}[1]{\markboth{\chaptername\ \thechapter~—~#1}} 
\renewcommand{\chaptermark}[1]{\markboth{{#1}}{}}
\renewcommand{\sectionmark}[1]{\markright{{#1}}} 
\fancyhead[CO]{\RL{\fontspec[Language=Hebrew]{SBL Hebrew}\rightmark}}
%\fancyhead[CE]{\RL{\leftmark}}
\fancyhead[CE]{\RL{\fontspec[Language=Hebrew]{SBL Hebrew}בגנות הלאומנות}}
\fancyhead[RE,LO]{\L{\thepage}}
\fancyfoot{} % clear all footer field
\fancyfoot[CO,CE]{}
\renewcommand{\headrulewidth}{0.4pt} 




%%% Titles %%%
\usepackage{titlesec, titletoc}

\usepackage{xcolor}

\newcommand{\inkAlight}{\addfontfeature{Color=BF0000}} % light red
	\definecolor{inkAlight}{HTML}{BF0000}
\newcommand{\inkAdark}{\addfontfeature{Color=800000}} % dark red
	\definecolor{inkAdark}{HTML}{800000}
\newcommand{\inkA}{\inkAlight}
	\definecolor{inkA}{HTML}{B80808}

%\newcommand{\inkB}{\addfontfeature{Color=BFBFBF}} % light grey
%	\definecolor{inkB}{HTML}{BFBFBF}
\newcommand{\inkB}{\addfontfeature{Color=808080}} % light grey
	\definecolor{inkB}{HTML}{808080}

\newcommand{\inkC}{\addfontfeature{Color=404040}} % dark grey
	\definecolor{inkC}{HTML}{404040}

\newcommand{\sectionfont}{\fontspec[Script=Hebrew]{Alef Bold}}
%\titleformat{\section}{\sectionfont\Large\color{inkC}}{\textbf{\LR{\thesection}}\sectionornament}{0.0cm}{}[\titleline{\color{pink}\titlerule}]
\newcommand{\sectionornament}{\hspace{0.5cm}{\symbolglyph{\inkA ⁘}}\hspace{0.5cm}}
\titleformat{\section}{\sectionfont\Large\color{inkC}}{\textbf{\LR{\thesection}}\sectionornament}{0.0cm}{}[\titleline{\color{inkB}\titlerule}]
\titleformat{\subsection}{\sectionfont\large\color{inkC}}{\textbf{\LR{\thesubsection}}\sectionornament}{0.0cm}{}[{\titleline{\color{inkB}\titlerule}}] % Solution from http://tex.stackexchange.com/questions/40088/strange-behaviour-of-titlesec-with-colored-titlerule


\newcommand{\transsection}[4]{
	\noindent
	{\Large #1: \hfill \Large #2}
	\\
	{\large #3 \hfill #4}
	\vspace{0.1cm}
	\hrule
	\vspace{1cm}
	\markright{#1}
}

\newcommand{\simplesection}[3]{
	\noindent
	{\Large #1}
	\vspace{0.1cm}
	\hrule
	\vspace{1cm}
	\markright{#1}
}

\newcommand{\begintranslation}{
	\vspace{1cm}
	\hrule
}

%%% Misc %%%
\usepackage{url, needspace, ifthen, endnotes, ruby, graphicx, tabulary, xltxtra, longtable, onimage, textpos}
%\usepackage[normalsections, normalmargins, normalfloats, normalindent, normaltitle, normalleading, normallooseness, normalbib, normalbibnotes]{savetrees}

%\newcommand{\middleline}{\begin{center}\Heb{———————}\end{center}}
%\newcommand{\middleline}{\begin{center}\Heb{\underline{~ ~ ~ ~ ~ ~ ~}}\end{center}}
%\newcommand{\middleline}{{\begin{center}\huge\fontspec{Bodoni Ornaments ITC TT}∂\end{center}}\vspace{-0.4cm}} FIXME!!!
%\newcommand{\middlelinenormal}{{\vspace{-2em}\begin{center}\huge\fontspec{Bodoni Ornaments ITC TT}∂\end{center}}} FIXME!!!
\newcommand{\middleline}{} % TEMPORARY
\newcommand{\middlelinenormal}{} % TEMPORARY

%\hyphenpenalty=5000
%\tolerance=99999
\tolerance=99999
%\widowpenalty=1000
%\clubpenalty=1000

\newcounter{edRemarkCounter}
\setcounter{edRemarkCounter}{1}
\newcommand{\edRemark}[1]{{\addfontfeature{Color=990099}[{\fontspec{Fontin}\addfontfeature{Color=990099}\textbf{\arabic{edRemarkCounter}}}:~#1]}\stepcounter{edRemarkCounter}}
%\newcommand{\edRemark}[1]{}

\newcommand{\mdash}{~—\ }

%%% Fonts and BiDi %%%
\usepackage{fontspec}
\usepackage{bidi}
\TeXXeTstate=1
\setmainfont[Mapping=tex-text,
	Script=Hebrew,
	AutoFakeSlant=-0.15,
	BoldFont={Rutz_OE Bold Pro}]
	{Rutz_OE Regular Pro}
%\setmainfont[Script=Hebrew]{Alef}
\setmonofont[Mapping=tex-text] {LMTypewriter10 Regular}
%\setmonofont[Mapping=tex-text, Scale=0.75]{Monaco}

\newfontfamily{\MerxSubstFont}[Script=Hebrew]{Vesper Pro}
\XeTeXinterchartokenstate=1
\newXeTeXintercharclass\MerxSubst
\XeTeXcharclass"2019=\MerxSubst
\XeTeXcharclass"201A=\MerxSubst
\XeTeXcharclass"201D=\MerxSubst
\XeTeXcharclass"201E=\MerxSubst
\XeTeXinterchartoks 0 \MerxSubst = {\begingroup\MerxSubstFont}
\XeTeXinterchartoks 255 \MerxSubst = {\begingroup\MerxSubstFont}
\XeTeXinterchartoks \MerxSubst 0 = {\endgroup}
\XeTeXinterchartoks \MerxSubst 255 = {\endgroup}


%\usepackage{xstring}
%\normalexpandarg
\newcommand{\C}[1]{{\LR{\fontspec{Junicode}#1}}}
\newcommand{\Gr}[1]{{\LR{\fontspec{Gentium}#1}}}
\newcommand{\Heb}[1]{{\RL{\normalfont#1}}}
%\newcommand{\bibHebrew}[1]{\BH{#1}}
\newcommand{\bibHebrew}[1]{}
\renewcommand{\emph}{\textbf}
%\renewcommand{\textbf}[1]{{\fontspec[Mapping=tex-text, Script=Hebrew]{Guttman Hodes}#1}}
