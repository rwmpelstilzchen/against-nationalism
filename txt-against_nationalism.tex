אנו מתנגדים לכל דגלי הלאום.
כל דגלי הלאום מתנגדים לנו.

בגנות הלאומיות

\section*{הקדמה}

מקורו של מנשר זה הוא בזמן מסויים ובמקום מסויים. הביא לכתיבתו מבצע „עופרת יצוקה” של מדינת ישראל ברצועת עזה, בין סוף 2008 לתחילת \edRemark{תיקנתי טייפו} 2009. עם הצטברות העדויות על מעשי הזוועה, ועליית מנין ההרוגים עד לכ־1,500, נערכו הפגנות גדולות ברחבי העולם, כאשר בבריטניה עצמה התפתחה תנועת מחאה משמעותית. תנועה זו התבטאה במחאות סדירות ברחובות הערים, גל של 28 התבצרויות באוניברסיטאות ברחבי המדינה, ובפעולות מזדמנות כנגד חברות שהיו לכאורה מעורבות במלחמה. למרבה הצער, נוספו להן פעולות עם גוון אנטישמי מובהק\footnote{חלונו של סופרמרקט טסקו מטרו בסטפני (\L{Stepney}) נופץ, ועל קירו רוססה הכתובת „מוות ליהודים”}).

חברי הפדרציה האנרכיסטית היו מעורבים במחאה במגוון דרכים, בין אם בהפגנות רחוב ובין אם בהתבצרויות. כאנרכיסטים, אנחנו מתנגדים למלחמה, למיליטריזם ולאימפריאליזם, ורואים בתנועה חזקה כנגד כוחות אלה מרכיב מהותי בבינלאומניות בפועל, כמו גם בתהליך עיצובו של הבטחון העצמי הנחוץ לתנועה עממית למאבק במדינה ובקפיטליזם.

למרות זאת, לא התרשמנו, בלשון המעטה, מהדרך בה התמיכה ב„התנגדות הפלסטינית”~— קרי חמאס, הג׳יהאד האיסלמי, גדודי חללי אל־אקצא ושאר מדינות־שבדרך באיזור~— הסתננה אל תוך הסלידה הלגיטימית שהרגישו הפעילים כשפצצות ופגזים ניתכים על ראשיהם של עזתים מן השורה. לקבוצות אלה~— שקראו לפלסטינים מן השורה להקריב עצמם למען האומה~— יש היסטוריה מובהקת של בלימה בנשק של מאבקי עובדים, דיכוי נשים, הומואים ולסביות, והפצת הלאומנות והאיסלאמיזם הריאקציונריות. עם התקדמות המלחמה, חשפו קבוצות אלה את עורן האמיתי בנסיונן להרוג ישראלים ללא אבחנה, בחיסול חשבונות אלים עם יריביהן, ובנסיון להשיג רווח פוליטי מפליטי המלחמה, בכך שמנעו מהם לקבל סיוע רפואי מעבר לגבול\footnote{חמאס מנע מעזתי לגשת לבית חולים שדה בצד הישראלי של מעבר ארז בסוף ינואר. ראה: (הכנס מקור עברי כאן או מקור הגארדיאן).}. פלסטינים פשוטים, שברחו בהמוניהם, כשהם מתעלמים מקריאותיהם של הארגונים החמושים ומעודדיהם המערביים להשליך עצמם אל אש הקורבן ולהצטרף ל„התנגדות”, הראו את טבעה האמיתי של אותה „התנגדות”.

כקומוניסטים אנרכיסטיים, תמיד התנגדנו ללאומנות, ותמיד קבענו את מרחקנו מן השמאל בכך שהתנגדנו בקול רם \edRemark{הורדתי מקף} לכל לאומנות~— כולל זאת של „אומות מדוכאות.” בעוד שאנו מתנגדים לדיכוי, ניצול ונישול מסיבות לאומיות, ומתנגדים לאימפריאליזם ולמלחמות אימפריאליסטיות, אנו מסרבים להצטרף לרוב השמאל וליפול לפח ההזדהות עם המפסידים, האנדרדוג ולהלל את „ההתנגדות”~— גם אם באופן „ביקורתי”~— אותו ניתן לזהות בתוך המעגלים הלניניסטים או הטרוצקיסטיים. אנו החזקנו בעמדה הזו בעבר לגבי צפון אירלנד, ואנחנו מחזיקים בה כיום לגבי ישראל/פלסטין.

אי לכך, על מנת לספק הקשר למסמך הנ״ל, ולהדגים כיצד השתמשנו בניתוח שלנו בפועל, אנו מצרפים אליו, כנספחים, שני מנשרים שחילקו פעילי הפדרציה האנרכיסטית במהלך המערכה נגד המלחמה בעזה, ובהם השתמשו גם חברים אנרכיסטים אחרים בממלכה המאוחדת, כגון סניפים של פדרציית הסולידריות (\L{Solidarity Federation}) האנרכו־סינדיקליסטית, כמו גם קבוצת התארגנו! (\L{Organize}) בצפון אירלנד. אנו מקווים כי תפוצתו של טקסט זה תהיה רחבה כמו זו של המנשרים המקוריים שלנו, שתורגמו לספרדית ולפולנית, והגיעו כל הדרך למרכז אמריקה, וכי הוא יהווה פתח לפולמוס ברחבי התנועה הקומוניסטית האנטי־מדינתית.

ספטמבר 2009

\section*{הלאום ולאומיות}

חיי היום־יום שלנו מעמתים אותנו שוב ושוב עם ההגדרה הלאומית שלנו. בין אם כשאנחנו משתמשים בדרכון, מחפשים עבודה, הולכים לבית החולים או דורשים דמי אבטלה, אנחנו נאלצים להתמודד האפשרויות והמגבלות הנובעות מהלאום שלנו. כשאנו נוסעים, מדליקים את הטלויזיה, קוראים עיתון או משוחחים אחד עם השני, השיוך של בני האדם לאחד מתוך כמה מאות סוגים של אנשים מתנוססת ברקע, ולעתים אף נמצאת בחזית. ההנחה היא שכולנו שייכים לקבוצה לאומית כלשהי, ושגם אלה שיכולים לטעון לכמה זהויות לאומיות עדיין מוגדרים לפיהן. החלוקה של אוכלוסית העולם ללאומים מובחנים, וארגון השלטון עליה בהתאם, נתונה מראש, ונראית מובנת מאליה כמו כל חוק טבע אחר. כשאנו מכריזים, למשל, שאנחנו בריטים, פולנים, קוראנים או סומאלים, נראה לנו שמדובר במרכיב חשוב בזהותנו, המשפיע במידה רבה על הדרך בה אנחנו מתייחסים לעולם סביבנו; מרכיב שנותן לנו בסיס משותף עם אנשים מסויימים אך מבדיל אותנו מאחרים.

הביורוקרטיה נותנת לאינטואיציה הזו משנה תוקף. הלאום הוא הסעיף הבסיסי ביותר בכל טופס שנמלא~— הוא שמגדיר לאילו זכויות והטבות יש לנו גישה, האם אנחנו נמצאים בתוך קהילת האזרחות שהלאומיות מניחה או מחוצה לה, ובסופו של דבר, האם אנחנו אדם לגטימי, „חוקי”. ואמנם, כשאנחנו נתקלים בביורוקרטיה, ההגדרות שהיא מקצה לנו מתמרות מעלינו: מגדר, לאומיות וגזע יותר מכולן. הן נראות לנו כחלק מובן מאליו מעצמנו, כמו צבע עיניים או סוג דם, ובדרך כלל איננו נוטים להטיל בהן ספק.

אך למרות שנראה כאילו הלאום מהווה תכונה יסודית שלנו ושל אחרים, הוא גם בעייתי ביסודו. ברמה אחת, הוא מגדיר את עצמו. עבור הביורקרטיה, הלאום פשוט קיים. או שיש לך את הפספורט הנכון, את הזכויות הנכונות, או שלא. אך ברמה אחרת, כמו בנוגע לכל שאלה חברתית, אנו מתמודדים כאן לא עם איזה פן „טבעי” של הקיום האנושי, אלא עם צורת התארגנות חברתית שיש לה גם מקור וגם הגיון פנימי משלה. מכאן שעלינו להתמודד עם השאלה, „מהי אומה?” (צריך לעבור ולהחליט מה זה אומה ומה זה לאומיות ומה זה עם ומה זה לכל הרוחות!!כליל)

התשובה הרגילה נובעת מההגיון הבריא: „עם” חולק בתרבות, היסטוריה, מוצא, תרבות, ערכים, ובדרך כלל, גם שפה; כל אלה הופכים אותו לאומה. אנשים בתוך האומה שייכים לקהילה אותה אינם חולקים עם זרים. מנקודת המבט הזו, העולם מורכב מאומות שכאלה~— כך היה תמיד וכך גם תמיד יהיה. אבל הלאומיות כאידיאולוגיה, בלי קשר ל„אומה” עליה אנו מדברים, היא אידיאולוגיה פוליטית, המתארת את הקשר בין „העם” לבין המדינה. מדינת־הלאום נתפשת כתוצר של הקהילה הלאומית, ומהווה את האופן בו קהילה זו מנהלת את ענייניה, וככלי של רצונה ושלומה הקולקטיביים~— לכל הפחות נראה שהמצב הרגיל, הטבעי ואף הרצוי הוא כשיש התאמה מוחלטת בין מדינה לבין אומה, וזוהי נקודת ההתחלה של כל שיתוף פעולה, פעילות עסקית והתארגנות בינלאומיים. הרטוריקה הזאת עומדת גם בבסיסן של מדינות שאינן טורחות להצדיק עצמן דרך דמוקרטיה ייצוגית.

אך כשמנסים לחשוף את התכונות שקובעות אילו אוספים של אנשים יחשבו לאומות ואילו לא, נתקלים בבעיות רבות. כשאנחנו מנסים להבהיר מה „בריטיות”, „גמביאניות” או „תאילנדיות” בעצם אומרות, אנחנו בצרות. המצדדים בלאומיות יעלו הצעות, אבל הן תמיד יהיו תכונות שטחיות ופופולריות, בין אם הן „כבוד”, „נאמנות”, „חירות”, „הוגנות”, או כל דבר אחר שבאופנה. הם יזכירו קומץ מוסדות לאומיים סמליים בעודם מתעלמים מרבים אחרים. לאומים ברמה הזו אינם דומים לאידיאולוגיות פוליטיות, אין להם מודל מובהק לארגון החברה, ואין אחדות עקרונית או מבצעית~— האחדות אותה מניחים היא שרירותית לחלוטין.

אין כללים נראים לעין \edRemark{הורדתי מקף} שמגדירים בצורה מובהקת מה הופך „עם” ללאום, בניגוד למחנות משותפים אחרים. הדרישות הנפוצות הן שפה ותרבות משותפות. אך תרבות משותפת זו קשה להגדרה, ולרוב ניתן למצוא הבדלים תרבותיים ניכרים באותה המידה בין אוכלוסיות בתוך אומות כמו שניתן למצוא הבדלים כאלה בין אומות. ההנחה היא כי לשני סינים בני ההאן ישנו המחנה המשותף של היותם „סינים”, ומכאן סולידריות טבעית על בסיס זה, גם אם הם דוברים שני 'ניבים' של סינית שלא מאפשרים להם לתקשר אחד עם השני. באותה המידה, ההבנה של הרצף בין „תרבות לאומית” היסטורית לבין מה  שקיים בשטח דורשת לא מעט קפיצות לוגיות~— למשל, כיצד ייתכן שתושב אתונה, המדבר יוונית מודרנית, שהתפתחה מיוונית אטית, מבטא את אותה התרבות שיצרה את האקרופוליס, שהייתה בעצמה תרבות יוונית נטולה אומה יוונית? לרוב, ה„אומה” הזאת כוללת בהכרח רבים שאינם חולקים בתכונות שאמורות להגדיר אותה: מיעוטים מקומיים, דוברי שפת מיעוט, מחזיקי תרבות מיעוט, מאמיני דת מיעוט, ואפילו מיעוטים „לאומיים”. העובדה, שניתן למצוא בתוך גבולותיהן הגיאוגרפיים של מדינות־לאום את אותו מגוון שניתן למצוא אם חוצים אותם, בולטת לעין במיוחד במדינות אפריקאיות פוסט־קולוניאליות רבות, במדינות בדרום־מזרח אסיה כגון אינדונזיה, ואפילו במקומות פחות אקזוטיים, כגון שוויץ.

אף־על־פי־כן, לאומנים נוהגים לצמצם את השאלה לנרטיב של „טבע האדם”, בו „עמים” פשוט אינם מסוגלים להתערבב מבלי ליצור עימות, כך ש„ההגדרה העצמית” של אומות דרך מדינותיהן הריבוניות הופכת לסדרם הטבעי של הדברים. החשיבה הזו לרוב שקועה בתורת־הגזע הפסבדו־מדעית, ומסתמכת על סיפורי „ככה זה” ומיתולוגיות נטורליסטיות מבולבלות. הטיעון כי „עמים” מוגדרים על־ידי האיבה שלהם כלפי „עמים” אחרים, אלא שהאיבה הזו נובעת מכך שהם „עמים” שונים, הוא טיעון מעגלי. ועדיין לא ניתנת סיבה ברורה לכך שלקבוצות מסויימות מגיע להקרא לאום, ולקבוצות אחרות לא. לאיבה בין איזורים מטרופוליטניים מסויימים ישנה שושלת ארוכה יותר מאשר לאותה איבה לאומית משוערת, אבל ממנה לא מניחים כי נובעת איזו גישה למעמד הלאום.

מעבר לכך, ישנן מדינות, כגון מדגסקר, כמו-גם איזורים שלמים, כגון שטחים גדולים באמריקה הלטינית, בהן ה„גזע” של האוכלוסיה הוא מעורב. במדגסקר, „העם המדגסקרי” הוא למעשה ערבוב מקומי של אוכלוסיות מתנחלים מאפריקה ואוסטרונזים. עקרון דומה חל גם על איזורים פחות אקזוטיים. „העם האנגלי” הוא למעשה ערבוב של גלי כיבוש והתנחלות, ו„התרבות הלאומית” שלהם אפילו יותר מעורבת מאשר ה„גזע” הגנטי שלהם.

הלאומיות, אם כך, היא דבר מוזר: הוא נמצא בכל מקום ומהווה חלק מן „ההגיון הבריא”, אבל לא ניתן להגדירו במדויק; הוא עקרון בסיסי לארגון כלל האוכלוסיה העולמית, אך מצד שני אינו מצליח לעמוד על שלו תחת בחינה דקדקנית.

אך לא תמיד היה כך. במשך רוב ההיסטוריה, אנשים לא השתייכו ללאום מסויים, או ללאומים צולבים, שהגדירו את אישיותם באופן כל־כך עמוק אך חמקמק, ובטח ובטח שלא התקיימה לאומנות שתתלווה אליו. בניגוד לדברי „ההגיון הבריא”, הגורס כי החלוקה הלאומית גילה כגיל האנושות, הרי שהמציאות היא אחרת לחלוטין. הלאומיות היא תוצר של העולם המודרני, ושזורה בתוך התפתחותו של סוג מסיום של חברה, שהשפעתו כיום כלל־עולמית ומוחלטת~— הקפיטליזם.



\section*{מוצא הלאומים}

הקפיטליזם ומדינת־הלאום המודרנית נוצרו באותו הזמן, ובאותו המקום: באירופה של המאות ה־16 עד ה־19. הווצרותם הייתה משולבת: האחד זרז את התקדמות השניה, ולהיפך. הקפיטליזם החל את דרכו בתקופה מסויימת ובאיזור מסויים לא באופן מקרי, אלא מכיוון שנתקיימו התנאים שנדרשו על מנת שיפרח: הוא נזקק לזירה מפוצלת של מדינות מתחרות, משובצות באינטרסים מרקנטיליסטיים, למרות שלמשך תקופה ארוכה הן עוד לא היוו את מדינות הלאום שאנו היינו מזהים ככאלה. לפיכך הוא נוצר באירופה, ולא באימפריה העותמנית, בסין המנצ׳ורית, או באחת מהאימפריות האחרות ששלטו בחלק ניכר מן העולם.

כמו הקפיטליזם, כך גם הרעיון של מדינת־הלאום המודרנית לא נוצר יש-מאין, אלא התפתח מהתנאים הקיימים. למרות זאת, הקפיטליזם כמערכת כלכלית מוחלטת, כמו גם עולם המורכב ממדינות־לאום ריבוניות, הן תופעה חדשה מאד, שחותמת היסטוריה ארוכה לש פיאודליזם ואימפריות. מדינת־הלאום המודרנית היא תוצאה של המהפיכות של המאה ה־18, שסימנו את דעיכת התקופה הפיאודלית ואת עלייתו של הקפיטליזם כמערכת חובקת־עולם. אבל התופעה הזו לא נפלה מן השמים ביום נפילת הבסטיליה\footnote{תושבי פאריס תקפו את מבצר הכלא הידוע לשמצה ב־1789 על מנת להשיג אבק שריפה, ובכך הציתו את המהפכה הצרפתית. המהפיכה הזו נתפשת לרוב כנקודת המפנה בין ריכוז הכח בידי האצולה לבין העברתו לידי המעמד הקפיטליסטי העולה.}, אלא טופחה ופותחה במקביל להווצרותו והבשלתו של הקפיטליזם.

החידושים הטכנולוגיים אשר לרוב מיוחסים להתפתחות המוקדמת של הקפיטליזם הניחו את היסוד להווצרותה של הלאומיות. הייצור וההפצה של ספרים מודפסים הייתה אחת התעשיות הקפיטליסטיות הראשונות. לאחר שהשוק ההתחלתי של אירופאים קוראי-לטינית מוצה, מילא המעבר לייצור ספרים המכוון אל הרובד המלומד בשפות איזוריות, שהיה עוד קטן אבל בתהליך גדילה, תפקיד חשוב ביצירת שפת אדמיניסטרציה ותרבות עלית, ולביסוס מה שינוכס כ„תרבות לאומית” במאות הבאות~— תפקיד שהיה חשוב במיוחד עבור האיזורים שהפכו לבסוף לגרמניה ולאיטליה. הרפורמציה\footnote{גל התהפוכות הדתיות והחברתיות ששטף את אירופה, מיסד את התנועה הפרוטסטנטיות והביא לירידת כוחה של הכנסיה הקתולית.}, שבעצמה חבה את הצלחתה לזו של תעשיית הדפוס, בשילוב עם קרנו העולה של מעמד הסוחרים במדינות אימפריאליסטיות~— שהצלחתו בסחר במוצרי חליפין היוותה ראש-גשר עבור יחסי-החברה הקפיטליסטיים באירופה~— הובילו לבסוף להקמת מספר מדינות שלא היו ממלכות שושלתיות וגם לא ערי-מדינה. הן לא היו מדינות־הלאום של הקפיטליזם המפותח, אבל בהחלט היוו צעדים משמעותיים בכיוון.

הבנקאים והסוחרים, שבעבר פעלו בשולי הכלכלות הפיאודליות, החלו לשחק תפקיד מרכזי יותר ויותר עם התפשטותן של האימפריות האירופאיות ברחבי העולם. הסחר שלהם בשלל שהניבו המושבות~— בשפע החומרי כמו-גם בעבדים~— הפך אותם לחיוניים לתפקוד הכלכלות שלהם, והדומיננטיות המתרחבת של האימפריאליזם האירופאי הגדילה את מספרם, ממונם, ואת חשיבותם הפוליטית. הצפיפות הגדולה שלהם בשבעה-עשר המחוזות של ארצות השפלה דרבנה את המרד שם, וכך הרפובליקה ההולנדית שנוצרה כתוצאה מכך, ב־1581, היוותה סימן לבאות. ההצלחה הכלכלית של סוחרי האימפריות השונות הובילה לזליגה של ההשפעה שלהם חזרה אל החברות מהן יצאו. בבריטניה, גידור האדמות הציבוריות\footnote{תהליך הניכוס לידיים פרטיות של אדמות המרעה משותפות של הכפר המסורתי, שהיה חשוב להתפתחות הקפיטליזם בבריטניה משתי בחינות: קודם כל, על־ידי כך שהניח את הבסיס להפיכת האדמה למוצר סחיר במקביל להתפתחויות מונחות-שוק בחקלאות, ובנוסף על־ידי נישול נתחים גדולים של האוכלוסיה, שנאלצו להפוך לעובדים שכירים.}, התפתחות התעשיה תחת לחצי הסחר, כמו גם הנצחון של תעשיינים בעלי-יוזמה בתחרות נגד יצרנים קטנים, יצרה מעמד עובדים מנושל, שלא הייתה לו ברירה אלא לעבוד עבור מעסיקים פרטיים~— במלים אחרות, הגורמים הללו הובילו ליצירת הקפיטליזם במלוא מובן המלה. התעשיין החליף את הסוחר כשחקן המרכזי של המעמד הבורגני.

במקביל לקיצו של העולם הפיאודלי, ולמעבר לעולם הממוקד באינטרסים של המעמד הקפיטליסטי העולה, המדינה הוגדרה מחדש. עתם של המלכים ונתיניהם התחלפה בעתם של ה„אזרחים”. תקופתן של השושלות המתחרות פינתה את דרכה לתקופה המודרנית, בה המדינות מתחרות. בעקבות המהפכות בצרפת ובאמריקה, התגבש רעיון הליברלי של המדינה, שהציב את הבסיס ללאומיות. הוא לא היה מתוכנן מראש~— לא היה צורך בכך, מכיוון שהוא לא קפץ אל המציאות מדמיונותיהם של אינטלקטואלים, אלא מתוך הצורך של החברה המעמדית המתפתחת ליצור את התנאים להמשכיותה-היא.

בהצהרת זכויות האדם והאזרח של 1795\footnote{המסמך שפרש את הזכויות האוניברסליות והיסודיות של אזרחי צרפת לאחר המהפיכה הצרפתית. ההבנה הרווחת הייתה שמקורם של זכויות אלה הוא בטבע האדם.}, הרעיון בא לידי ביטוי כך: 

„כל עם הוא עצמאי וריבוני, ואין זה משנה מה מספר החברים בו, או גודל השטח בו הוא מחזיק. לא ניתן להפר ריבונות זו.”

הבנה זו של תפקיד המדינה עמדה בניגוד מוחלט לאבסולוטיזם ששרר בתקופות מוקדמות יותר. עתה היה זה „העם” שהיה ריבון, במקום השליט בחיר האל. אבל בתקופה הזו לא הייתה הגדרה ברורה למהו „עם”. זו הייתה הגדרה מעגלית, שהייתה תלויה בשטח ובאוכלוסיה של מדינות קיימות, מכיוון שבשלב זה לא היו נסיונות רבים להגדיר אזרחות לאומית או „עמים” על בסיס לשוני, תרבותי או גזעי. כמעט תמיד היה מדובר בשאלה מעשית גרידא. „מדע” הלאומיות, וספרית ההגדרות הלאומיות שנתלוותה לו, התמלאו עד הקצהרק מאה שנה לאחר מכן. כאשר נעשו נסיונות הגדרה בשלב הזה, כמו למשל, במחצית השניה של המאה ה־18, הבינו אומות על בסיס היותן נשלטות על־ידי מדינות מסויימות. זה האופן בו הגדירה את האומות האינציקלופדיה הצרפתית, היצירה שנתפשת כמרכזת את החשיבה הנאורה של תקופת טרום-המהפכה, ושכרכיה פורסמו במהלך שנות ה־50 וה־60 של המאה הזו. לא הייתה כל דרישה לאחידות אתנית, לשונית או תרבותית~— עבור התיאורטיקנים של תקופת הנאורות, אומה לא הייתה יותר ממספר גדול של אנשים אשר היו תחומים בגבולות מוגדרים היטב, ואשר היו נתונים תחת אותו שלטון חוק.

המהפיכה בנתה מאומה זו של נתינים אומה של אזרחים~— האומה הפכה לכל אלה המסוגלים והמוכנים לקיים את תנאי האזרחות, המתבטאים דרך המדינה. תפישה זו נשמרת עדיין ברטוריקה~— אם לא בפרקטיקה~— של הלאומיות באחת האומות שנוצרה במהלך המהפכות של פאתי המאה ה־18: אמריקאים מוגדרים כאלה שמתגייסים למען „אמריקאיות” ושואפים להיות אמריקאים. עבור המהפכנים הבורגנים, הקהילה התיאורטית של „אזרחים”~— ולא משנה כיצד הגדירו אותה~— ייצג את הריבונות של האינטרס המשותף אל מול האינטרסים הצרים של הכתר, למרות שכך לא נראתה החברה המעמדית עליה הם שלטו, כמובן.

התפישה של לאומיות במונחים של יחודיות אתנית, תרבותית ולשונית הגיעה מאוחר יותר, במסגרת פולמוסים אינטלקטואליים לגבי מהות האומה, ולאילו „אומות”, כפי שיבחרו להגדירן, מגיע למצוא את ביטויין במדינת־לאום. תהליך ההגדרה של „עמים” תפס תאוצה מן הרגע שבו התבסס העקרון של המדינה כביטויו של „עם” ריבוני, לכל אורך המאה ה־19. ההוגה הפוליטי ג׳ון סטוארט מיל האלה לדיון את המוצא, השפה, הדת, השטח וההיסטוריה המשותפים כקריטריונים ללאום. אבל בעוד ההוגים התווכחו על מקורם של „עמים”, הנושא עדיין נתפש בעיקר במונחים מעשיים. השאלה אילו „עמים” צריכים להקים אומות הייתה שאלה של יכולת קיום, והאומות בנות-הקיימא היו לרוב כאלו שכבר היו קיימות. על מנת שאומה חדשה תחשב לזכאית למעמד זה, היה צריך להיות לה הבסיס הכלכלי או התרבותי שיאפשר לה לשרוד, כפי שקרה כשנוצרו איטליה וגרמניה בחצי השני של המאה ה־19. השאלה הקשה, כיצד להפוך אוכלוסיות לעמים ועמים לאומות, סיפקה תשובות עמומות בלבד, אבל התבססה בעיקר על גודל האוכלוסיה, הקשר שלה למדינה קודמת, הקיום של אליטה תרבותית בת-קיימא (כמו במקרה של הגרמנים והאיטלקים), והחשוב ביותר, קיומה של היסטוריה של התפשטות ולוחמה, לה היתרון שהיא מספקת אוייב חיצוני להתאחד כנגדו. אירלנד הייתה יוצאת-מן-הכלל, בכך שהכילה תנועה לאומית שהקדימה את זמנה~— למעשה, היא היוותה את המודל הארכיטיפי ללאומיויות שנוצרו בשנים מאוחרות יותר, כמו של ההודים והבאסקים. אף-על-פי-כן, התייחסו בביטול ליכולת הקיום של התנועה הזו בשל בעיות מעשיות.

ועדיין, רוב ה„עמים” שבסופו של דבר הקימו „אומות” עדיין לא ראו עצמם במונחים לאומיים, ולא ראו שום עיוות מוסרי בכך שנשלטו על־ידי אליטה שדיברה שפה אחרת, בעיקר מכיוון שבעולם מלא בניבים מקומיים מובחנים ובאנאלפבתיות, לא היו שפות לאומיות מאוחדות. אפילו תפקידן של השפות „הרשמיות” לא דמה כמעט בכלל למעמדן של שפות לאומיות בימינו. המטרה שלהן הייתה מעשית בלבד, ולא היה להן כל קשר ל„תודעה לאומית.” כך היה במשך זמן רב. באנגליה, למשל, שפת האליטות התפתחה מאנגלו-סקסונית, ללטינית, לנורמנית ולבסוף לשעטנז של צרפתית נורמנית ואנגלו-סקסונית שהיה האנגלית המוקדמת. שפת האליטות לא הייתה רלוונטית לנתינים חסרי ההשכלה. גם בתקופות מאוחרות יותר התמונה הייתה זהה~— בשנת 1789 רק 12\% מהאוכלוסיה בצרפת דיברה צרפתית „תקנית”, וחצי מהם כלל לא דיברו צרפתית מכל סוג. למרות שקיומה של תרבות אליטה דוברת-איטלקית הייתה חיונית ליצירתה של מדינה איטלקית במאה ה־19, רק 2.5\% מהאוכלוסיה „האיטלקית” דיברה שפה זו בעת האיחוד~— האוכלוסיה בכללה דיברה מגוון רחב של ניבים, ובמקרים רבים לא ניתן היה להבין ניב אחד בעזרת האחר.

אמנם היו נסיונות מועטים ומוגבלים לספר כיצד נוצרו לאומים שונים במאות הקודמות~— למשל, הסיפורים שהסתובבו בצרפת של המאה ה־16 על מוצאם של הצרפתים (קרי, האליטות הצרפתיות) מן הפראנקים ומטרויה~— אך אלה היו מוגבלים למעגלים קטנים של משכילים, ונועדו לתת תוקף לזכויות מלכותיות ו/או אריסטוקרטיות, זכויות שהגנו עליהן בתכיפות, בהצלחה ובפופולריות רבות יותר על־ידי פניה למצוות אלוהיות או לתקדימים רומאיים. סיפורים אלה נוצרו כתוצאה מהתקשרויות בתוך מעגל קטן של משכילים שחלקו בשפה ובזכויות-יתר מוסדיות משותפות. זו הייתה נקודת התחלה ללאומיות של המאות הבאות. לא היה בכך כל ביטוי ל„תודעה לאומית” מודרנית ופופולרית, שהרי לא היה להם הכוח המניע הפופולרי של לאומיות, התפישה כי המדינה אמורה להביע את טובתה של האומה בכללותה, וכי יש לבנות את האומה הזו ברמה העממית. כאשר הלאומיות המודרנית נעשתה דומיננטית, ניסיונותיהן של השושלות הישנות ליישב את עצמן איתה היו הרות אסון: למרות שקייזר וילהלם השני הודר יותר ויותר ממרכז הבמה במהלך מלחמת העולם השניה, הוא הציג את עצמו כראשון מבני האומה הגרמנית, ולכן נבעה מכך אחריות מסויימת לעם הגרמני ולאינטרס הלאומי~— ומכאן הגיע למסקנה כי הוא מעל באחריות הזו, אותה המסקנה שהביאה לכך שויתר על כסאו. בשנים שלפני כן, בהן זכותו של הקייזר הייתה בלתי מוטלת בספק, והוא לא היה חייב דין וחשבון לאף אחד, היה ראיון זה נשמע מגוחך.

ובעוד המאה ה־19 התקדמה, כך גם הרעיון שלכל העמים ישנה הזכות להגדרה עצמית, בלי קשר ליכולת הקיום שלהם. הלאומנים האיטלקים והפילוסוף ג׳וספה מציני הציגו את הנוסחה: „לכל אומה~— מדינה, ורק מדינה אחת לכל אומה” כפתרון ל„בעיה הלאומית”. דרך החשיבה הזו התבהרה לקראת סוף המאה, באותו הזמן בו הלאומיות כרעיון נעשתה נפוצה בקרב ההמונים. ריבוי התנועות הלאומיות ותנועות „השחרור הלאומי” בסוף המאה ה־19 הוא מרשים~— לידתה של הציונות לצד התנועות ההודית, הארמנית, המקדונית, הגאורגיאנית, הבלגית, הקטלנית ועוד רבות אחרות, התרחשו בתקופה הזו, אם כי שאלה אחרת היא עד כמה הייתה להן אחיזה בקרב האוכלוסיה הכללית. אמנם היו קבוצות אתניות או לשוניות בתקופות קודמות, שראו עצמן כמובחנות במובן כלשהו משכנותיהן, אבל המעבר מהבחנה זו להכרה בצורך במדינת־לאום לכל קבוצה היה תופעה חדשה. מעבר לכך, ה„מכנה המשותף” בו השתמשו על מנת להגדיר את האומה, כפי שהבינו אותה, גם הוא נוצר על־ידי העידן החדש~— הדפוס, החינוך המודרני, התחבורה והתקשורת הובילו למחיקת הוריאציות הלשוניות המקומיות וליצירת תרבות ציבורית שאפשרה נקודת אחיזה לרעיון האומה. זה לא היה אפשרי בתקופות קודמות, בהן לא הייתה קיימת עדיין התשתית הזו לשבירת החומות התרבותיות, שהיו לפעמים אפילו בין כפר לשכנו. השפה הלאומית, שלרוב היוותה תנאי מוקדם ללאומיות פעילה, הייתה המצאה חדשה יחסית. היא דרשה אוריינות גבוהה יותר, ניידות אנושית, ואת פירוק היחסים החברתיים הקרתנים, הפיאודליים, כפי שכבר ראינו. בניגוד לפנטזיות של הלאומנים, שרואים את השפה המשותפת כאותו קשר בסיסי עליו מתבססת מדינת־הלאום, הרי שלמעשה הייתה השפה המשותפת תוצר של התפתחות המדינה המודרנית.

בעשורים האחרונים של המאה ה־19, הרעיון שלכל „עם” ישנה זכות מוסרית למדינת־לאום משלו היה כבר מושרש היטב. שאלת המעשיות, ששיחקה חלק כל־כך חשוב בפולמוסים קודמים בנושא, כבר לא עלתה על סדר היום. זו הייתה זכותם של „עמים”, כיצד שלא יוגדרו, לקבל מדינה משל עצמם. שליטה של לאום אחד בלאום אחר הייתה מוקצה מחמת מיאוס (בתיאוריה, לפחות~— לאימפריאליזם הייתה לוגיקה משל עצמו). זו הייתה התקופה בה ההגדרה האתנית והלשונית של „האומה” התגברה על כל ההגדרות שקדמו לה. מדינות־הלאום האימפריאליסטיות המתחרות של הקפיטליזם המודרני לבשו את צורתן הסופית, ותנועות שדגלו בהתנגדות להן ובהנתקות מהן תפשו את הפעילות שלהן, כמו-גם את המטרות הסופיות שלהן, במונחים של יצירת מדינות־לאום חדשות.

התפתחותה של הלאומיות המודרנית הייתה קשורה קשר הדוק לעובדה, שהמדינה הקפיטליסטית המודרנית, שאוכלוסייתה המנוצלת משכילה יותר, דורשת יותר מאזרחיה מאשר שקודמותיה הפיאודליות דרשו מן האיכרים הפאסיביים. נדרש לה כוח חברתי מאחד, ולשלב את מעמד העובדים לתוך שיטת הממשל; לפיכך, היא הייתה זקוקה לנאמנות של האוכלוסיה, ולא יכלה להסתפק בחוסר-התנגדות תבוסתני, כפי שהיה עם האיכרים. המצאת הפטריוטיות מילאה את הצורך הזה. המודעות והנאמנות ל„ארץ אבותינו” התפתחה ונעשתה נפוצה בין מדינות־הלאום האירופאיות במהלך השליש האחרון של המאה ה־19. התפתחות המונח „פטריוטיזם” מהווה משל לתהליך כולו. ה„פטריה” (\L{Patrie}), „המולדת” שמהווה את הבסיס למונח, התייחסה לאיזור מוצא מקומי בלבד, ולפני המהפכה הצרפתית, לא היו לכך שום השלכות לאומיות. בסוף המאה ה־19, לעומת זאת, היא כבר הייתה לקהילה המדומיינת של האומה, שדרשה את השתתפותם של ההמונים. בעזרתה של תורת-הגזע הפסבדו-מדעית, שהפכה לתחליף כה חשוב לפאגאניות של הילידים כצידוק לנישול אימפריאליסטי של כל מיני אוכלוסיות מקומיות, נולדה אידיאולוגית העליונות הלאומית.

העקרון הזה הגיע לשיאו במלחמת העולם הראשונה, ובתקופה שמיד לאחריה. הלאומנות השוביניסטית של סוף המאה ה־19 הפכה לאידיאולוגיה של מלחמה כוללת, של טבח ממוכן בין גושים לאומיים מגוייסים. כל היבט של חיי היום-יום נוכס על־ידי „האינטרס הלאומי”~— סכסוכים פנימיים נאלצו לפנות מקומם לעליונות האומה, כמו גם להשרדות, שכן כל מדינה לוחמת טענה שהיא לוחמת מלחמת מגן. בעקבות מרחץ-הדמים הקפיטליסטי הזה, מפת אירופה צוירה מחדש בקווים לאומיים. נעשה נסיון להביא את האידיאל של „כל אומה~— מדינה” מן הכוח אל הפועל, ו„האידיאל של ווילסון”\footnote{וודרו ווילסון, נשיא ארה״ב בסוף המלחמה, מילא חלק חשוב בהצגת הלאומיות וההגדרה העצמית כדרך הטובה ביותר לארגון היחסים הבינלאומיים.} של „הגדרה-עצמית לאומית” הפך למציאות גיאופוליטית. פירוק האימפריה האוסטרו-הונגרית למדינות־לאום חדשות היה נסיון לפתור את בעיית „האומות המדוכאות”. זה לא עבד, מסיבות שהן אינטגרליות ללאומיות~— המדינות החדשות האלה לא היו אחידות, ובעצמן כללו מיעוטים חדשים.

ברגע שמקבלים את עקרון „ההגדרה העצמית” של „עמים”, אין לזה סוף. מכאן נובעת התפוצה הרחבה של מיעוטים לאומיים עויינים ברחבי העולם, ומעטות הארצות שלא נכוו מהתהליך: הרי העקרון היסודי של הלאומיות הוא שלקולקטיבים של בני אדם ישנה הזכות להגדרה עצמית דרך האומה „שלהם”, אבל בסופו של דבר, אין שום דרך להגדיר אילו קבוצות של אנשים הן „אומות” ואילו לא. עם הזמן, קבוצות קטנות יותר ויותר עוטות את האדרת הזו.

אם כך, ללאומיות יש מקור והיסטוריה ארציים לגמרי. הכוח שלה נובע מהדרך בה היא מוצגת כדרכו הטבעית של העולם, ומן ההנחה שחלוקה לאומית והגדרה לאומית הן חלק טבעי מחיי האנושות~— שכך תמיד היה וכך תמיד יהיה. אנרכיסטים מחזיקים בדעה מאד שונה. אותה תקופה היסטורית שיצרה את מדינת־הלאום ואת הקפיטליזם יצרה גם גוף שנעלם מן התיאורים הלאומיים~— המעמד המנושל של העובדים עבור שכר, שלו אינטרסים המנוגדים לאלה של מדינת־הלאום הקפיטליסטית~— הוא מעמד העובדים. מעמד זה, שמחויב לפעול למען האינטרסים שלו כנגד ההון אינו „עם”, אלא תנאי קיום בתוך הקפיטליזם, ולכן הוא מתעלה מעל לגבולות הלאומיים. העויינות הזו הובילה ליצירת תפישות-עולם שקוראות תיגר על העולם הקפיטליסטי, ומציגות עולם אחר לחלוטין. תפישת-העולם שלנו, קומוניזם אנרכיסטי, היא אחת מהן.



\section*{מדוע מתנגדים האנרכיסטים ללאומיות?}

אנרכיסטים השייכים למסורת המאבק המעמדי (במלים אחרות, קומוניסטים אנרכיסטים), כמו הפדרציה האנרכיסטית, לא רואים את העולם במונחים של עמים לאומיים מתחרים, אלא במונחים מעמדיים. אנחנו לא רואים עולם המלא באומות נאבקות, אלא במעמדות נאבקים. הלאום הוא מסך עשן, פנטזיה שמסתירה את המאבק הבין-מעמדי שמתקיים בתוך האומות ובין האומות. לתפישתנו, אין אומות אמיתיות, אבל יש מעמדות אמיתיים עם אינטרסים משל עצמם, ויש להפריד בין המעמדות הללו. לפיכך, אין „עם” יחיד בתוך „האומה”, ואין „אינטרס לאומי” משותף המאחד אותם.

קומוניסטים אנרכיסטים לא מתנגדים ללאומיות רק כאשר היא משתלבת עם גזענות ועם קרתנות חשוכה. אמנם, הלאומיות בהחלט מעודדת תופעות כאלה, וגייסה אותם לטובתה במהלך ההיסטוריה. מכאן שמרכיב עיקרי בפוליטיקה האנרכיסטית היא ההתארגנות כנגדם. אלא שהלאומיות אינה זקוקה להם על מנת לפעול. הלאומיות יכולה להיות ליברלית, קוסמופוליטית וסובלנית, ולהגדיר את ה„אינטרס המשותף” של „העם” בדרכים שלא דורשות „גזע” יחיד. אפילו האידיאולוגיות הלאומניות ביותר, כגון הפאשיזם, יכולות לחיות בכפיפה אחת עם חברה מעורבת, כפי שהיה עם התנועה האינטגרליסטית הברזילאית\footnote{תנועה פאשיסטית בברזיל, שהתמודדה עם חוסר ההצלחה שלה בגיוס ההמונים על בסיס גזעי על־ידי קידום את הססמה: „איחוד כל הגזעים וכל העמים”, בעודה משתמשת באותה הרטוריקה כנגד קומוניזם, ליברליזם וכיו״ב שבה השתמשו מקבילותיה האירופאיות.}. הלאומיות משתמשת במה שעובד~— היא לוקחת על עצמה כל תכונה שתאפשר לה לכבול את החברה יחדיו מאחוריה. במקרים מסויימים היא משתמשת בגזענות גסה, במקרים אחרים היא נוהגת באופן מתוחכם יותר. היא מתמרנת את מה שקיים למטרותיה שלה. במדינות מערביות רבות, רב-תרבותיות רשמית היא מרכיב עיקרי במדיניות האזרחית, ונוצרה לאומיות רב-תרבותית הנובעת ממנה. הרב-תרבותיות הרשמית עצמה הופכת להיות „התרבות הלאומית”, ומאפשרת השתלבות של „אזרחים” אל תוך המדינה מבלי להזדקק לחד-תרבותיות גסה. גם לו הייתה הרטוריקה הלאומית של המדינה הקפיטליסטית מהפתוחות, הסובלניות והאנטי-גזעניות ביותר, עדיין היו האנרכיסטים מתנגדים לה.

הם היו ממשיכים לעשות זאת מכיוון שבמהותה, הלאומיות היא אידיאולוגיה של שיתוף-פעולה בין-מעמדי. מטרתה ליצור קהילה מדומיינת בעלת אינטרסים משותפים, ובכך להסתיר את האינטרסים האמיתיים, החומריים, של המעמדות מהם מורכבת האוכלוסיה. „האינטרס הלאומי” הוא נשק המופעל נגד מעמד הפועלים; מטרתו לגייס את תמיכת הנשלטים באינטרסים של השולטים. הגיוס האידיאולוגי, ולעתים הפיזי, של האוכלוסיה בסדר-גודל המוני בשם איזו תכונה לאומית משותפת ומרכזית עמד במרכז כל המלחמות של המאות העשרים והעשרים ואחת עד כה~— מרחץ הדמים בעירק,  שהתרבות הדמוקרטית המערבית היוותה תירוץ לו, כמו גם חיזוק המדינה בשם ההגנה על המסורת הדמוקרטית והחופשית הבריטית או האמריקאית כנגד הטרור האיסלאמי הן דוגמאות רלוונטיות מאד לימינו.

בסופו של דבר, ההתנגדות האנרכיסטית ללאומיות מבוססת על עקרון פשוט: למעמד העובד ולמעמד המעסיק אין שום דבר משותף. זו לא סתם ססמה: מדובר בראייה נכוחה של העולם בו אנו חיים. עויינות מעמדית היא חלק אינהרנטי של הקפיטליזם, וימשיך להתקיים בין אם קבוצות אינטלקטואליות ופוליטיות ידונו בקיומו או חוסר-קיומו, ובין אם לאו. מעמד אינו קשור למבטא שלך, למנהגים הצרכניים שלך, או לצבע הצווארון שלך, כחול או לבן. מעמד העובדים~— מה שנקרא לפעמים הפרולטריון~— הוא המעמד המנושל, המעמד שאין לו הון, שאין לו שליטה על תנאי המחיה הכלליים שלו, ושאין לו ממה לחיות אלא מהיכולת שלו לעבוד עבור משכורת. לעובד יכולים להיות בית ומכונית, אבל הוא עדיין יצטרך למכור את יכולתו לעבוד למעסיק בתמורה לכסף שהוא זקוק לו על מנת לחיות. יש לו אינטרס ספציפי, אובייקטיבי, וחומרי: לקבל יותר כסף מהמעסיק שלו על פחות עבודה, ולשפר את חייו ואת תנאי העבודה שלו. האינטרסים של ההון הם הפוכים: להוציא מאיתנו יותר עבודה עבור פחות שכר, לעגל פינות ולחסוך בעלויות, על מנת להגיע לרווח גדול יותר ועל מנת שהכסף יהפוך לעוד יותר כסף מהר יותר ובצורה יעילה כמה שיותר. המאבק המעמדי הוא התחרות בין האינטרסים האלה. אפילו מקומות עבודה לא-יצרניים מעוצבים על־פי החוקים הללו, מכיוון שהם העקרונות היסודיים של החברה הקפיטליסטית. האינטרסים של ההון מתבטאים על־ידי בעלי הכוח, וגם הם מחוייבים להגן על האינטרסים האלה על מנת לשמור על כוחם, בין אם הם הבעלים של הון פרטי, המנהלים שמחליטים בשם אותו ההון, או המדינה שנחוצה על מנת לקדש ולהגן על הרכוש הפרטי ועל זכות הקניין.

„האינטרס הלאומי” הוא פשוט האינטרס של ההון בתוך גבולות המדינה. זהו האינטרס של הבעלים של החברה, שבתורם אינם מסוגלים אלא לבטא את הצרכים הבסיסיים של ההון: להמשיך ולצבור או לחדול. בענייני פנים, מטרתו של ההון היא לביית את אלה בתוך החברה שעלולים להיות עויינים לו~— מעמד העובדים. העויינות הזו, שהיא אינהרנטית לקפיטליזם, היא כזו שהאנרכיסטים רואים בה את הפוטנציאל להתקדם אל מעבר לקפיטליזם. עלינו להאבק על האינטרסים שלנו כדי להשיג את צרכינו כויתורים מטעם ההון. התהליך הזה קורה גם אם לא מבססים תיאוריות מורכבות סביב לו. עובדים בסין או בנגלדש, המתבצרים במפעלים ופורעים נגד המדינה, לא עושים זאת דווקא בגלל שהם נתקלו באיזו תורה מהפכנית, אלא בגלל שתנאי המחיה שלהם מחייבים אותם לעשות כן. באופן דומה, הסולידריות קיימת לא בגלל שאנשים הם טובי-לב, אלא בגלל שהסולידריות משרתת את האינטרסים שלהם. לבעלי ההון יש את המדינה: את החוק, את בתי המשפט ואת בתי־הכלא. לנו יש רק אחד את השני. לבדנו אנו יכולים להשיג מעט מאד, אבל ביחד אנחנו יכולים להפריע לתפקודו התקין של הקפיטליזם, וזהו נשק חזק מאד. אמנם, מאבקים מעמדיים לרוב אינם טהורים, ויכולים להיות מעורבים בהם אינטרסים פלגניים ודעות קדומות מסוגים שונים. לפיכך זהו תפקידם של קבוצות מהפכניות וארגונים אנרכיסטיים במקום העבודה להלחם בנטיות האלה, לתרום להתפתחותה של התודעה והפעלתנות המעמדית, ולהשלים את התהליך שבו קוראים תיגר על הפילוגים על־ידי מאבק משולב שמתקיים בתוך מאבקים גדולים ומשמעותיים.

המעמד השליט מודע מאד לסוגיות הללו, ופועל במודע למען האינטרסים שלו. הסולידריות היא הדבר היחידי שאנו יכולים להחזיק מעל לראשיהם, ובגלל זה המדינה עסוקה מאד בלגרום לנו לפעול כנגד האינטרסים שלנו. הלאומיות היא אחת מהנשקים הכי חזקים שלהם בהקשר הזה, ולכן היא מילאה תפקיד היסטורי מאד חשוב. היא מאחדת אותנו עם אוייבנו, ודורשת מאיתנו שנבטל את האינטרסים שלנו כעובדים בפני אלה של האומה. היא מובילה לביות של מעמד העובדים, מובילה לכך שעובדים מגדירים עצמם דרך הלאום, ורואים פתרונות לבעיות שלהם במונחים לאומיים. זה אינו תהליך סופני, כפי שאנחנו כבר יודעים~— הרי הנסיבות יכולות לדחוף אנשים לפעול למען האינטרסים שלהם, ודרך התהליך הזה, רעיונותיהם מתפתחים ומשתנים. אם לתת דוגמה דרמטית מההיסטוריה, עובדים מרחבי העולם צעדו למלחמה בה נקראו לטבוח אחד בהשני ב־1914, ואז עברו לשאת את נשקיהם לעבר אדוניהם בגל בינלאומי של שביתות, מרידות, התקוממויות ומהפיכות מ־1917 והלאה.

אף על פי כן, הלאומיות היא עשב שוטה שיש לעקור מן השורש. זו אידיאולוגיה של ביות. זהו נשק כנגדנו. זו קרתנות מאורגנת, שמטרתה לפלג את מעמד העובדים לאורך קווים לאומיים~— כאשר בעצם, כרכיב בתוך המערכת הכלכלית, המעמד הזה הוא בינלאומי.

בסופו של דבר, גם אם היינו מניחים בצד לרגע את ההתנגדות העקרונית והתיאורטית שלנו ללאומיות, עצם הרעיון של הגדרה-עצמית לאומית בעלת משמעות בעולם המודרני היא אידיאליזם לשמו. אומות אינן יכולות להיות עצמאיות כשהן נמצאות תחת שליטתו של שוק קפיטליסטי עולמי, וגורמים המקבעים את המסגרת הפוליטית שלהם במונחים של החזרת הריבונות הלאומית כנגד הקפיטליזם העולמי, כמו פשיסטים מודרניים וחבריהם לדרך, מחפשים למעשה מעין תור זהב בלתי ניתן להשגה שקדם לקפיטליזם המודרני. העולם המודרני הוא משולב: אי אפשר להפריד בו בין „שיתוף-פעולה” בינלאומי לבין עימות, ושני אלה מבוטאים בו על־ידי מוסדות בינלאומיים כגון האו״ם, ארגון הסחר העולמי, הבנק העולמי, האיחוד האירופי, ברית נאט״ו, וכיו״ב. הפנטזיה הלאומית היא ריקה מתוכן, באותה המידה שהיא ריאקציונרית. האנרכיסטים מכירים בזאת בהתנגדות שלהם. עוד נחזור לנקודה הזו.

לפני שנמשיך, נקדים תרופה למכה, ונפריך „ביקורת” כושלת נפוצה: אנחנו לא מייצגים חד-תרבותיות. אנחנו לא מכוונים לחברה אנרכיסטית שבה כל המגוון התרבותי האנושי יצטמצם לאיזה סטנדרט אפרורי. וכיצד נוכל לעשות זאת? הערבוב הטבעי של התרבות שם ללעג את הפנטזיות הלאומיות. גושים לאומיים אף פעם אינם חסינים להשפעות תרבותיות חיצוניות, והתרבות מתפשטת ביניהם ומתערבבת כל הזמן. הרעיון של תרבויות לאומיות עצמאיות, שמקודם על־ידי לאומנים, הוא מיתי לחלוטין. מה שאנו מציבים מנגד לכך הוא את החליפין החופשיים ברעיונות בחברה קומוניסטית חופשית, ללא מדינות, חליפין שהוא תוצר טבעי של המאבק כנגד המדינה והקפיטליזם.

ההתנגדות הקומוניסטית האנרכיסטית ללאומיות צריכה להאמר בקול ובברור. לא נרשה לעצמנו להציג בינלאומנות מזויפת. בינלאומנות אינה שיתוף-פעולה של אומות קפיטליסטיות, או של מעמדי עובדים לאומיים, אלא ביקורת יסודית על עצם הרעיון של האומה ושל הלאומיות.



\section*{השמאל ו„השאלה הלאומית”}

המראה של קבוצות שמאל, התומכות בארגונים ומדינות ריאקציונריות, הוא מאד נפוץ בימינו. הוא נמצא תחת ביקורת ממקורות רבים, ומסיבות שונות. אנחנו מבינים לחלוטין את הגועל שעולה באנשים רבים למראה סוציאליסטים בעיני-עצמם שמעודדים ארגונים כגון החמאס, שקוראים „כולנו חיזבאללה” בהפגנות „נגד המלחמה”, ושתומכים בשלטונות המדכאים מאבקי עובדים, כולאים ומוציאים להורג עובדים, מדכאים נשים ורודפים אחר הומואים ולסביות. אבל לאופן החשיבה שמאפשר לזה לקרות יש אילן יוחסין מפואר. הדרך שבה תנועות מרקסיסטיות התאימו עצמן ללאומיות, ובמקרים רבים אף פעלו כמיילדות של תנועות לאומיות ומדינות־לאום, שהיו דוחות באותה המידה שהיו אחיותיהן המערביות, נמצא בבסיסה של הלאומיות ה„אנטי-אימפריאליסטית” הנוכחית. הכרת הקשר שבין תנועות הפועלים לבין הלאומיות היא חיונית על מנת להבין את „מלחמות השחרור הלאומי” המודרניות ואת התגובה אליהן.

מרקס עצמו, כמו בנושאים רבים, לא סיפק עמדה נחרצת, שניתן לומר עליה בבטחה שהיא „שלו”. המניפסט הקומוניסטי, למרות שהיווה תוכנית לא-קומוניסטית בעליל, נסגר בקריאה המפורסמת: „פועלי כל העולם, התאחדו!” בכך ביטא את ההתנגדות הבינלאומנית לביותו של מעמד העובדים על־ידי הלאומיות. באותו הזמן, מרקס ואנגלס חלקו בראיה הלאומית-ליברלית הרווחת באותה התקופה, בה תהליך בנית המדינות היה תהליך של איחוד, ולא של פירוק. אמירה מפורסמת של אנגלס הייתה שהוא לא רואה את הצ׳כים שורדים כעם עצמאי מהסיבה הזו. למשך זמן מה, תמכו מרקס ואנגלס ב„שחרור הלאומי” של פולין (ובהמשך, בתנועה לעצמאות, שמוביליה היו אריסטוקרטים) מסיבות אסטרטגיות~— על מנת להכות ברוסיה האוטוקרטית, ולפי תפישתם, להגן על ההתפתחות הקפיטליסטית במערב אירופה, ומכאן על קיום התנאים המוקדמים לתקומת הסוציאליזם שם. הגישה שלו לאירלנד גם היא הושפעה משיקולים טקטיים דומים. דיון בצדקתה של גישה זו בתקופתנו, בה קיים כבר קפיטליזם עולמי מפותח, יהיה אקדמי בלבד, ומעבר לגבולותיו של מנשר זה. אך ברור למדי כי במובנים רבים מרקס שיקף את התפישות שהיו נפוצות בלאומיות הליברלית בין תחילת המאה ה־19 ועד לאמצעה, כפי שתוארו קודם לכן.

הדרישה השמאלית להגדרה עצמית לאומית כזכות הפכה לעדכנית באותו הזמן שהיא הפכה לעדכנית בכלל העולם הפוליטי, ופולמוסים לגבי „השאלה הלאומית” הרעידו את האינטרנציונאל השני, כאשר העימות בשאלה זו בין לנין לבין המרקסיסטית ילידת פולין רוזה לוקסמבורג הפך לידוע לשמצה. עמדותיו של לנין היו סותרות באופן אופייני, למרות שבמתארן הכללי הסתמכו טענותיו על יסודות דומים לאלה של מרקס בנושא זה~— יש לתמוך בשחרור הלאומי כל עוד הוא מקדם את התפתחות מטרותיו של מעמד העובדים ואת התנאים המוקדמים לסוציאליזם. למרות הייסוד הטאקטי והאמביוולנטי הזה, הזה, הבולשביקים היו חד-משמעיים בתמיכתם ב„זכות האומות להגדרה עצמית”, לאחר שהאינטרנציונאל השני העביר החלטה שתומכת בכך.

רוזה לוקסמבורג התנגדה לתפישה הזו. לוקסמבורג הכירה בכך שסוגיית „העצמאות הלאומית” הייתה ענין של כוח, לא של „זכויות”. עבורה, הדיון על „זכויות” של „הגדרה עצמית” היה אוטופי, אידיאליסטי, ומטאפיזי~— נקודת הייחוס שלו לא הייתה הניגוד החומרי בין המעמדות, אלא עולם המיתוסים הלאומיים הבורגני. היא הייתה קולנית במיוחד בנושא זה כאשר התווכחה עם הסוציאליסטים הפולנים, שראו בעמדתו הקודמת (והטאקטית) של מרקס ברכת דרך תמידית ללאומיות שלהם עצמם.

אף-על-פי-כן, הבולשביקים הם אלה שתפסו את המושכות ברוסיה, והנהיגו את מהפכת-הנגד במדינה הזו. לאחר מלחמת האזרחים, הובילה התמיכה שלהם ב„זכות האומות להגדרה עצמית” לכמה ניסיונות משונים ב„בניית-אומות”, שהקבילו לנסיונות של וודרו ווילסון והסכם ורסאי באירופה, שנים מספר קודם לכן \footnote{הסכם ורסאי חתם את המלחמה~— כאשר בנות הברית הכתיבו את התנאים, ומפת אירופה צוירה מחדש על בסיס עקרונות הלאומיות, היכן שהיה אפשר.}. יצירת „יחידות אדמיניסטרטיביות לאומיות” עבור „אומות” לא-רוסיות שונות בתוך ברית המועצות שהוכרזה לא-מכבר נבעה מדעות קדומות של ביורוקרטים סובייטיים, ולא מתוך איזה רצון ללאומיות מצד האוזבקים, הטורקמנים או הקזחים. עם ריסוק המהפיכה הרוסית על־ידי המשטר הקפיטליסטי-מדינתי תחת שלטון הבולשביקים, שחיסלו או ספגו לתוכם גם את אמצעי הארגון שמעמד העובדים יצר לעצמו וגם את המהפכנים שהגנו עליהם (למשל האנרכיסטים), הרי שהשאלה ממילא איבדה מתוכנה: השיקול היחידי של הבולשביקים בשלב זה היה קידום כוחם שלהם. כמו יריבתה המערבית, ברית המועצות ניצלה את הרטוריקה של „ההגדרה העצמית” ו„העצמאות” על מנת להרחיב את מעגל ההשפעה שלה.

ובכל זאת, העקרון שללאומים ישנה זכות אינהרנטית להגדרה-עצמית כנגד „דיכוי לאומי” השתרש כהגיון בריא בקרב תנועת הפועלים, כפי שעשה באוכלוסיה הכללית.



\section*{מאבקי שחרור לאומי}

לאחר שגיבשו את כוחם במהלך מלחמת האזרחים, לבשה מדיניותם של הבולשביקים את אותה גלימה לאומית צפויה. ב־1920, הבולשביקים סיפקו תמיכה לתנועה הלאומית הבורגנית של קמאל פשה בטורקיה, בשל מכת המחץ שנצחונה תסב לאימפריאליזם הבריטי. הייתה זו הדוגמה הראשונה לניצול התמיכה ב„מאבקי שחרור לאומי” „אנטי-אימפריאליסטיים” למען האסטרטגיה הגיאופוליטית בולשביקית. עבור מעמד העובדים בטורקיה, היה זה אסון. התוצאה הייתה מחיצה נמרצת של שביתות והפגנות בידי הרפובליקה הטורקית החדשה. בדומה, גם הקומינטאנג~— התנועה הלאומית הסינית~— קיבלה תמיכה סובייטית, דבר שהוביל לטבח בעובדים מרדניים בשנחאי. המעמד השליט החדש ברוסיה תמך באוייבים אלה של מעמד העובדים בשם ההגנה על המהפיכה. ביניהם היו שצבעו את הקפיטליזם המדינתי הלאומני שלהם בצבעים קומוניסטיים, אבל גם אלה ייצגו תנועות להקמת מדינת־לאום בת-קיימא, עם מעמד עובדים מנוצל, וכלכלת-שוק קפיטליסטית (-מדינתית).

להתפתחות הזו הייתה השפעה עמוקה על השמאל ברחבי העולם. היא ביססה עוד יותר את מקומה של התמיכה ב„מאבקי שחרור לאומי” כחלק בסיסי של „ההגיון הבריא” של תנועת העובדים. זה היה נכון לא רק לזנים השונים של סוציאליסטים-מדינתיים~— הטרוצקיסטים, המאואיסטים והסטליניסטים~— אלא הייתה לכך השפעה גם על כמה אנרכיסטים.

עבור הסטליניסטים, עם פוליטיקה שממילא הכילה בדמה את הלאומיות, היו „מאבקי שחרור לאומי” כלי לשיבוש תמרוני ארה״ב, לטובתה של ברית המועצות~— שתמכה במאבקים כאלה חומרית או פוליטית במסגרת קידום מטרותיה האימפריאליסטיות. עבור המאואיסטים, ואלה שהושפעו על־ידי המהפיכה הקובנית, ניפוץ האימפריאליזם המערבי על־ידי השחרור הלאומי היה תנאי הכרחי על מנת לאפשר לתנועות האיכרים והפועלים במדינות האלו לפתח במהרה את הכלכלה שלהם~— לדבריהם, למען רווחת האוכלוסיה. עבור הטרוצקיסטים, פותחו כל מיני מסגרות היסטוריות שהסבירו מדוע האימפריאליזם היה, כמו בתיאורו של לנין, צורתו הגבוהה ביותר של הקפיטליזם, ומדוע היה למטרה הסוציאליסטית אינטרס בתבוסה של האימפריאליזם בידי כוחות השחרור הלאומי.

לכל אלה הצטרף בשנות השישים של המאה ה־20 גל של אידיאולוגיות מוכוונות העולם השלישי, ששיקף במידה רבה את הכשלון הנסיון ליצירת תנועה מהפכנית מתוך אי-השקט הרווח, שהתקיים בזמן התפרקותו של הקולוניאליזם המערבי. על בסיס כתביו של לנין נבנתה התפישה, כי מעמד העובדים המערבי נשלט על־ידי „אצולת העבודה”, שהתבססה על סחיטת העושר מקורבנות האימפריאליזם. מכאן, התקווה לסוציאליזם נמצאה ב„הגדרה העצמית” של עמים לא-מערביים. התמיכה בתנועות אקזוטיות, על בסיס התנגדותן ל„אימפריאליזם”, שצומצם לאימפריאליזם האמריקאי, בשילוב רלטיויזם מוסרי, ממשיכה עד ימינו אנו. ניתן לראות אותה בהתלהבות הרבה שבה מתקבלים ריאקציונרים מוסלמים בקרב שמאלנים מערביים.

התפישה הזו היא כמובן שגויה וריאקציונרית, בהציבה את העימות הלאומי לפני העימות המעמדי. אבל מעבר לכך, בתקופה שלאחר מלחמת העולם השניה, לשמאל הפוסט-קולוניאליסטי היה מונופול של ממש על תנועות השחרור הלאומי עצמן. הסטליניזם כבר מזמן התאים עצמו ללאומיות נושאת-דגלים, ובמקרים רבים לא ניתן היה להבדיל בין הרטוריקה שלו לבין פאשיזם ממש. השמאל לקח חלק מרכזי בתנועות ההתנגדות לפאשיזם באירופה במהלך מלחמת העולם השניה, מה שאיפשר לקבוצות האלה לעטות את האדרת הלאומית לאחר השחרור, ולנהוג כנציגים המרכזיים של ה„רצון” החופשי של האומה. דוגמה בולטת לכך היא התפקיד המרכזי שהיה ל־\L{EAM-ELAS}\footnote{\L{EAM}~— „צבא השחרור הלאומי של העם היווני”. \L{ELAS}~— „חזית השחרור הלאומי”, הזרוע הצבאית של ה־\L{EAM}. שתיהן נשלטו על־ידי המפלגה הקומוניסטית היוונית הסטאליניסטית, שניסתה להגיע לשלטון לאחר התבוסה הגרמנית במלחמה.} בהתנגדות היוונית במהלך המלחמה. פעיליה לא בחלו בעריפת ראשיהם של פעילים אנרכיסטיים או של תנועות העובדים, ואפילו לא ברציחת יריבים בתוך תנועת ההתנגדות עצמה. בתקופה שלאחר המלחמה, החיבור הזה בין שמאלנות לבין פטריוטיזם עיצב את המראה השמאלני של תנועות שחרור לאומי קולוניאליות שונות, והפכה את הלאומיות למרכיב מרכזי בשמאל, ואת השמאל למיילדת של תנועות לאומיות ברחבי העולם.

לצערנו, אין לאנרכיסטים חסינות מתפישות כאלה. אנרכיסטים רבים הצליחו להציג מאבקים למען „שחרור לאומי”~— כלומר, מאבקים למען צורה אחת של המדינה על חשבון צורה אחרת~— במונחים של המאבק נגד הדיכוי, שמהווה את מטבע הלשון הבסיסית של הפוליטיקה האנרכיסטית. הטיעון שלהם הוא, שכמו שאנרכיסטים מתנגדים לדיכויים השונים הפושים בעולמנו: ניצול מעמד העובדים, ודיכוי הנשים ומיעוטים מגדריים, מיניים ואתניים, עלינו גם להתנגד לדיכוי של אומה אחת בידי אומה אחרת. יש לכך ביסוס מסויים במסגרת המסורת האנרכיסטית הקלאסית, כמו באמירתו הידועה-לשמצה של באקונין: „לכל לאום, קטן או גדול, ישנה הזכות הבלתי-מעורערת להיות עצמו, לחיות לפי הטבע שלו. הזכות הזו היא תוצאה הגיונית של עקרון החירות הכללי.” דוגמה יותר עדכנית באה מתוך „החברה והטבע” (\L{Society and Nature}) של מוריי בוקצ׳ין: „אף ליברטריאן שמאלני... אינו יכול להתנגד לזכותו של עם כבוש להקים עצמו כיישות אוטונומית~— בין אם בהתאגדות שיתופית... ובין אם במדינת־לאום המבוססת על אי-שוויונות היררכיים ומעמדיים.”

באופן דומה, שמאלנים רבים מערבבים בין ההתנגדות למלחמה אימפריאליסטית לבין התמיכה בשחרור לאומי, או לפחות מטשטשים את ההבדלים מספיק על מנת שהבלבול יהיה בלתי-נמנע. כך הם הופכים את הזוועה המוצדקת שמרגישים כתגובה למלחמות כאלה על פיה, ועוברים מעמדה כנגד המלחמה לעמדה בעד המלחמה~— כפי שהיא מנוהלת על־ידי הצד החלש, האנדרדוג. ההיסטוריה מלאה בדוגמאות לכך: ממפגינים נגד מלחמת וייטנם שדקלמו את שמו של מנהיג צפון קוריאה, הו צ׳י מין, ועד מפגיני שמאל שהכריזו „עכשיו כולנו חיזבאללה” במהלך ההפגנות נגד הפצצת לבנון בידי מדינת ישראל.

עלינו להתנגד לתמיכה הזו במדינה החלשה, האנדרדוג, או המדינה לעתיד. אין שום מהות בסיסית להתנגדות הלאומית, וכוחות השחרור הלאומי אינם מתעלים איזו רוח לאומית מדוכאת. הכוחות האלה הם כוחות מאורגנים אמיתיים, עם מטרות משלהם~— להקים סוג מסויים של מדינה נצלנית, הנשלטת על־ידי פלגים מסויימים. האומה איננה יצור בראשיתי שדוכא, אלא נרטיב שנבנה על־ידי המדינה הקפיטליסטית במהלך התפתחותה. אמנם המבנה האימפריאליסטי הופך לחלק ממערך הניצול של מעמד העובדים בשטח המושפע ממנו, אך לפעול לארגון מחדש של מערך הניצול הזה עבור מדינה „ילידית” מהווה מטרה ריאקציונרית. כפי שכבר ראינו, ההגיון הלאומי הוא ריאקציונרי במהותו, בכך שהוא מאגד מעמדות שונים לקולקטיב לאומי אחד. מעבר לכך, במונחים מעשיים לגמרי, לעקרון הלאומי אין סוף. המדינות החדשות, ה„עצמאיות”, תמיד מכילות מיעוטים מהם נשללת „ההגדרה העצמית”. בנוסף, צורות הניצול שמנחילים שליטים „ילידים” לאחר תום מאבקים לשחרור לאומי אינם שונים ברמה המעשית מהשיטות של ה„זרים”. העובדים בצפון קוריאה מדוכאים על־ידי מדינה ילידית „קומוניסטית”, שהיא אכזרית במידה דומה לדיקטטורות הפאשיסטיות של המאה העשרים; עובדים בוייטנאם מנוצלים על־ידי כלכלה קפיטליסטית מונחית-ייצוא; עובדים בזימבבווה, ששוחררה מן האימפריאליזם הבריטי, הפכו לקורבנותיו של שלטון גנגסטרי „ילידי”. לא קשה למצוא דוגמאות נוספות. כל הארצות האלה חוות מאבק מעמדי בעצימות רבה או מועטה. מאבק מעמדי הוא חלק ממארג הקפיטליזם, גם בקפיטליזם-מדינתי עריץ במודל הסובייטי, וכך יהיה בלי קשר לשאלה האם המעמד השליט הוא „ילידי” או לא.

מעבר לכך, כאשר הן משוחררות מן הדיכוי הלאומי של הקולוניאליזם המערבי, המדינות „המשוחררות” הללו מגלות עצמן כמסוגלות ליזום מלחמות אכזריות משלהן. הדוגמה של וייטנאם היא מאלפת. מיד לאחר האיחוד-מחדש ב־1976, שבא בעקבות נסיגת כוחות ארה״ב ב־1973, וייטנאם לקחה חלק בסדרת מלחמות לכל רוחבה של תת-היבשת הודו-סין. זו החלה במלחמה טריטוריאלית אכזרית עם החמר רוז׳, שעלו לשלטון לאחר ההפצצות הנוראיות של ארה״ב בקמבודיה, מלחמה שהסתיימה בכיבוש המדינה הזו על־ידי חיילים וייטנאמיים. כל זה הוביל לדומיננטיות של וייטנאם באיזור, שנתמכה על־ידי האימפריאליזם הסובייטי. לאוס הייתה בפועל למדינת חסות של וייטנאם, שהחזיקה שם בסיסי צבא, והכריחה את ממשלת לאו לנתק את קשריה עם סין. ב־1979, כתוצאה מהמלחמה בין וייטנאם לבין מדינת החסות הקמבודית שלה, כמו גם תקריות גבול שונות ותביעות טריטורליאליות סותרות, פלשה אליה סין, דבר שהביא לעשרות אלפי הרוגים ולהרס רב בצפון וייטנאם.

„השחרור” של אומות מעול האימפריאליזם הוביל למעגלים נוספים של מלחמה בחלקים אחרים של העולם, כשתנועות לאומיות רבות מן המאה העשרים פעלו כנגד מדינות חדשות, פוסט-קולוניאליות, במקום המעצמות המערביות. סרי לנקה היא דוגמה של החיבור בין הצלקות המתמשכות של האימפריאליזם המערבי לבין משחקי הכוח של מעמדות שליטים קהילתיים, שהוביל לסחרור מתמשך של מלחמה ואלימות אתנית-לאומנית, בו תנועות לאומיות מתחרות משליכות את מעמדות העובדים „שלהם” אחד נגד השני שוב ושוב.

המנהל האימפריאליסטי הבריטי בסרי-לנקה כונן מערכת של ייצוג קהילתי בועדה המחוקקת של האי מאז אמצע המאה ה־19. דבר זה יצר עויינות בין המיעוט הטמילי לבין הרוב הסינהלי, שנמשכת עד עצם היום הזה. אחרי מלחמתה העולם השניה, משניתנה זכות הצבעה שווה לכולם, ולאחריה עצמאות למדינה עצמה, הטמילים~— שזכו עד כה לייצוג-יתר בממשלה~— נדחפו מעמדות-היתר שלהם, מה שגרם להחרפת הרצון של המיעוט הטמילי להתנתק מצד אחד, ומצד שני לאפליה גוברת כלפיו. הכיבוש של איזורים דוברי-טמילית על־ידי הממשלה הסינהאלית, ההכרזה על סינהלית כשפה הראשמית, והאיסור על ייבוא של ספרים, עיתונים ומגזינים בטמילית מאיזורים טמיליים בהודו העמידו את התשתית החברתית לקיומן של קבוצות חמושות טמיליות ולמלחמת האזרחים בסרי לנקה.

התרחבות שורותיהם של ארגונים חמושים, כגון הנמרים הטמיליים (\L{LTTE}~— \L{Liberation Tigers of Tamil Eelam}) הידועים לשמצה, ניזונה מעוולות אמיתיים שנגרמו לטמילים, במיוחד לאחר הפוגרומים של יולי השחור, שנת 1983, בהם נטבחו מאות טמילים. למרות זאת, האשליה, שהנמרים הטמיליים, עם כל „הטעויות האיומות” שעשו, עדיין מהווים את הכוח להגנה עצמית של הטמילים\footnote{כפי שטרוצקיסטים, למשל, נוהגים לטעון.} נמוגה כאשר נזכרים שבין המטרות הראשונות כנגדן נלחמו היו ארגונים טמיליים יריבים, לאומיים וקומוניסטיים, כמו \L{Tamil Eelam Liberation Organization}, שלמעשה נמחק מעל פני המפה על־ידי הנמרים הטמיליים ב־1986. לאחר שהנמרים הטמיליים הפכו לממשלה בפועל בכמה איזורים טמיליים, הם פנו אל המיעוטים החדשים~— מוסלמים סרי לנקים~— וביצעו בהם טיהור אתני על־ידי פינוי, איומים ולבסוף טבחים המוניים, כולל הרג במכונות יריה של גברים, נשים וילדים שנכלאו בתוך מסגד. כמויות לא-מבוטלות של עובדים סינהליים שנשארו באיזורים שנשלטו על־ידי הנמרים הטמיליים זכו לגורל דומה. לאומיות, גם זו של „לאומים מדוכאים”, אינה יכולה להציע אלא סיבובים חוזרים ונשנים של אלימות ועימותים, כתוצאה מחלוקת מעמד העובדים לפי קווים לאומיים, והקרבתם על מזבח „האינטרס הלאומי”, בין אם זה של מדינה קיימת או של מדינה בהתהוות.

היעדרו של האימפריאליזם המערבי אינו מביא שלום, ושחרור לאומי אינו מוביל להגדרה-עצמית, שאינה אפשרית בעולם הקפיטליסטי. זה נובע מטבעה האימפריאליסטי של מדינת־הלאום.



\section*{כל מדינות־הלאום הן אימפריאליסטיות}

ל„אימפריאליזם” היסטוריה ארוכה, והוא התבטא בצורות שונות ומשונות, שהתאימו להתגלגולתה של המדינה ושל החברה המעמדית בתקופות שונות. מכיוון שהמלה מתארת פרוייקטים שונים של מדינות שונות בתקופות מגוונות, עלינו להסביר למה הכוונה במסגרת שלנו: החברה הקפיטליסטית המתקדמת. האימפריה הרומאית הייתה שונה מהאימפריה הבריטית, ואילו האימפריאליזם הנוכחי שונה משניהם. זה לא אומר שאיננו יכולים לזהות מהו „אימפריאליזם”, אך בכל זאת, עלינו להגדיר בצורה מדוייקת יותר את התופעה אותה אנו מתארים.

כוחן של האימפריות הקלאסיות של העולם העתיק נבע מכיבוש שטחים ומגיוס המשאבים שלהם. הרצף שבין שליטת המדינה באדמות וכוחה של האימפריה הפך את האימפריאליזם שלהן לאבטיפוס של התופעה~— הצורה הבסיסית והשקופה ביותר שלה.

לאור זאת, נראה ש„מדיניות החוץ” של מדינות הלאום הקפיטליסטיות של היום היא יצור שונה בתכלית. אבל בעולם המודרני, האימפריאליזם משובץ בתוך ההתנהלות של המדינות לפחות באותה המידה שהיה זה נכון עבור כל תקופה אחרת בהיסטוריה. תפקודו וטבעו של האימפריאליזם השתנה ביחד עם הארגון הכלכלי של החברה שלקח בה חלק. כפי שצורת המדינה בחברת עבדים חקלאית היא שונה מזו של חברה קפיטליסטית מפותחת, כך גם האימפריאליזם של אותה המדינה. אבל למרות מגוון השינויים שהעולם עבר מאז, המדינה היא עדיין השחקן העיקרי באימפריאליזם של היום. ייתכן שזו נראית כמו הערה מוזרה בעולם שבו המעצמות השולטות הן דמוקרטיות ליברליות, השולחות אינספור עסקנים לאינספור פגישות, פסגות, פורומים וארגונים בינלאומיים. אף-על-פי-כן, האימפריאליזם חיוני לחלוטין לתפקודן של חברות קפיטליסטיות, ולא ניתן להפריד את ההצלחה שלו מן ההצלחה של המעצמות המובילות.

הלחצים של הקפיטליזם שינו את האימפריאליזם שקדם לו והזין אותו. גל ההשקעות הספקולטיביות שבקע מאירופה מאז שנות החמישים של המאה ה־19, כשההון חיפש כיצד להרוויח, הוביל להגברת הפעילות האימפריאליסטית, ומדינות נדחפו לשמור ולהסדיר את האינטרסים של ההון בתוך גבולות השליטה שלהם. התהליך הזה התגבר אף יותר לאחר שנות השבעים של אותה המאה. השלטון הבריטי הישיר בהודו, שהוקם בתגובה למרד שסיכן את האינטרסים הבריטיים שם, הוא דוגמה מוקדמת אחת (לפני כן הודו נשלטה על־ידי חברה פרטית בריטית), והבהלה לאפריקה משנות השמונים עד למלחמת העולם הראשונה ייצגה את המעבר הברור מן „האימפריאליזם הלא-רשמי” של העשורים הקודמים אל שיטה של שלטון ישיר, בו המעצמות האימפריאליסטיות חילקו את העולם ביניהן.

כפי שאנו כבר יודעים, המערכת הזו התפרקה בעקבות מלחמת העולם השניה, שהתחילה את תהליך הדה-קולוניזציה שהתקיים ברובו בחציה השני של המאה העשרים. למרות זאת, המנגנון המהותי דרכו מדינות פועלות למען ההון שבארצן על־ידי \edRemark{תיקנתי טייפו} תמרונם של אי-שיוויונים גיאו-פוליטיים נשאר מרכיב מרכזי בעולם הקפיטליסטי.

המדינה חייבת לפעול על מנת לקדם את האינטרסים של ההון~— מה שנקרא לרוב „האינטרסים העסקיים”~— בתוך הארץ עליה היא מופקדת. בתוך אותה ארץ, היא מטפחת את הקפיטליזם, היא מקדשת את חוקי הקניין שהקפיטליזם זקוק להם על מנת להתקיים, הוא פותח מרחבי צבירה עבור ההון, הוא מציל את ההון מנטיותיו להשמדה עצמית (גם לקול מחאתם של בעלי הון מסויימים), ומנהל את המאבק המעמדי על־ידי שילוב של כפיה וספיגה: הוא יכול ואכן מנפץ שביתות, אבל הוא גם נותן לאיגודי עובדים תפקיד רשמי בניהול כוח-העבודה, וכך יוצר שסתום דרכו ניתן לשחרר את לחץ המאבק המעמדי בצורה מבוקרת. המדינה היא „בעל ההון הקולקטיבי”~— היא הערב והחתם של השיטה הקפיטליסטית.

התפקיד הזה תקף גם עבור „מדיניות החוץ.” המדינה נושאת ונותנת על מנת לספק לחברות מקומיות גישה למשאבים, השקעות, סחר והזדמנויות התפשטות בחו״ל. ההצלחה של תהליך זה מביאה רווחים חזרה ארצה, ובכך שהיא מעשירה את העסקים בה ואת „הכלכלה הלאומית”, המדינה מבטיחה את הבסיס החומרי לכוחה שלה: היא מגדילה את כמות המשאבים והעושר שלה, ואת יכולתה להקרין את כוחה. מכאן שהיא יותר מאשר בובה של „אינטרסים תאגידיים”: למעשה, היא מהווה בעלת עניין בזכות עצמה.

באותו הזמן, על המדינה למנוע מעצמה להשלט על־ידי מדינות אחרות: עליה לגייס את המשאבים שלה~— הצבאיים, הדיפלומטיים, התרבותיים והכלכליים~— על מנת לשמור על מעמדה הבינלאומי. היא מקיימת מאבק מתמשך~— בין אם זה בשיחות מסביב לשולחן הדיונים עם „שותפים בינלאומיים” בנוגע למדיניות הסחר, ובין אם זה בעימותים צבאיים ב„נקודות חמות” ברחבי העולם~— על מנת לדאוג לקידום „האינטרס הלאומי” בחו״ל ולשמור עליו בבית. היא מקדמת את האיטרסים האלה על־ידי שימורם והשפעה על אי-שוויונים שקיימים בתוך הקפיטליזם לאורך המרחב הגיאוגרפי. אי-הסימטריות הללו מתבטאות כיום בדרך כלל דרך תופעות כגון מונופולים איזוריים, סחר לא-שווה, הגבלת זרימת ההון, והשפעה על דמי שכירות מונופוליסטיים. האימפריאליזם מתבטא בהפעלת ההבדלים הללו לטובת הכלכלה של המדינה~— קרי, ההון שבתוכה. זהו התפקוד הרגיל של הכלכלה העולמית, וניתן לראותו למשל בגיוס האמריקאי של קרן המטבע הבינלאומית וארגון הסחר העולמי לטובתן של התעשיות הפיננסיות האמריקאיות, או בתמרונים הסיניים באפריקה שמדרום לסהרה. כל מדינה חייבת לקחת חלק במערכת הזו של איזוני-כוח משתנים תמידית ללא  קשר למניעיה, שכן מדינות שאינן מצליחות לדחוף מעליהן את הלחצים האלה או לנהל אותם ימצאו עצמן נתונות לחסדיהם.

כך למלחמה יש תפקיד ברור. התערבויות אימפריאליסטיות יכולות לעתים לנבוע מתוך רצון להישגים ניתנים לכימות, כגון ניצולו של משאב מסויים. אך בדרך כלל השאלה היא שאלה של אסטרטגיה גיאו-פוליטית ואיגוף גושי-כוח אחרים על מנת לשמר כוח איזורי או בינלאומי. ההתייחסות אל המשאבים היא לרוב במונחים אסטרטגיים, ולא ניצול פשוט. אם כל שרצתה ארה״ב היה לנצל את הנפט העירקי במפרץ הפרסי, הרי היה הרבה יותר זול וקל להשאיר את סדאם בשלטון ולשאת ולתת איתו על הסכמי גישה. השאלה הייתה שאלה של שליטה במשאב האסטרטגי הזה, ומכאן הפלישה לעירק. השליטה בנפט המזרח-תיכוני, לו זמן-מדף ארוך יותר מזה של מאגרים יריבים, תאפשר לארה״ב שליטה בפועל בכלכלה העולמית, ובפרט בכלכלות של סין, רוסיה, יפן ואירופה, עם תעשיות הפיננסים והייצור המתחרות שלהן.

בדומה לכך, אין לכיבוש אפגניסטן קשר לניצול של משאב מסויים, אלא ליכולת לשלוט בנקודה אסטרטגית בקווקז, ולהקרין לתוך מעגלי ההשפעה של רוסיה ושל סין. הבריטים והרוסים כבשו את אפגניסטן מסיבות דומות. מלחמת וייטנאם הייתה עלולה לסכן את הצבר-ההון קצר הטווח, אבל בכל זאת היא היוותה חלק מאסטרטגיה אימפריאליסטית רחבה יותר, שהייתה אמורה לתרום לאינטרסים של ההון האמריקאי, בכך שתדאג לשמור על תפקידה הבכיר של ארה״ב בהנהגה הגלובלית וש„העולם החופשי” יישאר בטוח להשקעה ולניצול.

למרות כל זאת, כאשר הם נתקלים במנהגים האלה, שמאלנים רבים מסיקים מסקנות תמוהות. כחלק מההגיון של לתמוך במאבקי שחרור לאומי, והצורך למצוא איזה שליח לתמוך בו, שמאלנים ירבו לעודד את משטריהן של מדינות שנתונות למזימותיו של האימפריאליזם המערבי. אבל ל„דיכוי לאומי” אין שום קשר עם מאבק מעמדי, והתמיכה במשטרים, שעוסקים באופן פעיל בדיכוי העובדים „שלהם” וברדיפה אחר מיעוטים, כפועל יוצא של המדיניות ה„אנטי-אימפריאליסטית” שלהם, היא ריאקציונרית לחלוטין. תמיכה זו נובעת גם מחוסר ההבנה של האימפריאליזם, שהרי הוא תוצאה של המערכת הקפיטליסטית העולמית. אחרי הכל, למדינות והון לאומי הנמצאים ביחסי חולשה אל מול המעצמות, יש גם יחסים אסימטריים עם גורמים אחרים. ל„קורבנות” של האימפריאליזם המערבי יש סדר יום, וגם מדיניות אימפריאליסטית משלהם. ודאי שלאירן ולונצואלה, למשל, יש כאלה~— לוונצואלה בקידום האינטרסים שלה על־ידי הרחבת מעגל ההשפעה שלה ברחבי אמריקה הלטינית, וכך גם לאיראן בעירק, לבנון, אפריקה ובמקומות אחרים.

האימפריאליזם הוא יותר מהשפעה הבוקעת מחופן מעצמות גדולות, מדכאת ארצות קטנות יותר ומרחיבה את השפעתן ברחבי העולם. אמנם ישנן דרכים אימפריאליסטיות שהצליחו יותר מאחרות, אבל האימפריאליזם הוא בדמה של מדינת־הלאום. גם אם מדינה תשאף להשאר „מתורבתת” ולהמנע מן התחרות והעימותים שהן פועל יוצא של האימפריאליזם, היא תאלץ להגן על עצמה מפני ניסיונות לניצול החולשה שלה על־ידי גורמים אחרים, תוך שימוש בשיטות ישירות יותר או פחות. לפיכך, מדינות שיש להן יכולת מועטה יותר להקרין את עצמן מתיישרות עם אלה שיש להן יכולת רבה יותר, תוך שימוש בהגיון, שאפילו ילד יכול להבין.



\section*{אחרי הלאומיות}

נשארה שאלה אחת, שאנו נשאלים לעתים קרובות: אם אנרכיסטים לא מיישרים קו עם השמאל בתמיכתם במאבקי שחרור לאומי, ובדרישה להגדרה-עצמית לאומית, במה אנחנו כן תומכים? מהי האלטרנטיבה שלנו?

במובן מסויים, ניתן לדחות את השאלה הזו על הסף. יש דברים רבים שאנו לא תומכים בהם מסיבות עקרוניות, ושאי אפשר לדרוש מאיתנו שנציג להם אלטרנטיבה. הסירוב לתמוך בדבר שהוא ריאקציונרי בפועל במטרותיו עדיף מאשר „לעשות משהו” שמנוגד לעקרונות היסודיים שלנו. הלאומיות אינה יכולה להציע דבר פרט לסיבובים חוזרים של עימות, שמספרם ומידת חומרתם רק תלך ותגדל, כנראה, לאור התגברות התחרות על משאבי האנרגיה האוזלים של העולם. כאשר מנסחים עימות במונחים לאומיים~— ומבינים אותו כעימות בין לאום מדכא לבין לאום מדוכא~— מעמד העובדים מפסיד בכל מקרה.

בינלאומנים מכירים כבר את התגובה ההיסטרית בה מתקבלות ההתערבויות שלהם בפעילויות פוליטיות. בעיני רבים, „ההתנגדות” לחירחורי המלחמה האימפריאליסטים נמצאת מעל לכל ביקורת~— היריבים של פרוייקטים אימפריאליסטיים מסויימים מהווים „התנגדות” בלבד, נטולת כל יוזמה, מטרות או יעדים משל עצמה כפלג קפיטליסטי. התמיכה השמאלנית ב„התנגדות הפלסטינית”, למשל, פועלת לפי ההגיון הזה~— היא ניתנת גם לקבוצות כגון חמאס, שמדכא מאבקי עובדים, מנתץ משמרות שובתים באיומי נשק, מדכא נשים, מתאכזר להומואים ולסביות ואף הורג אותם. אבל כל זה כאילו לא היה, ברגע שבו מכלילים את חמאס בתוך „ההתנגדות”, שהרי לבקר את „ההתנגדות” זה ייהרג ובל יעבור. להביא פרספקטיבה מעמדית לסוגיה הזו, להפיץ את העובדה שארגוני השחרור הלאומי מתנהגים בדיוק כמו הארגונים הקפיטליסטיים שהם, ומגינים על האינטרסים של הקפיטליזם, המדינה או המדינה בהתהווות כנגד כל בדל איום של מאבק מעמדי של העובדים, כל זה שקול בעיניהם לתמיכה באימפריאליזם. לפי ההגיון הזה, סירוב ליישר קו עם פלג אחד זהה ליישור קו עם הפלג השני.

הבעיה היא שהנטיה לראות את העולם במונחים של לאום, במקום במונחים מעמדיים, נטועה עמוק בתוך הפסיכולוגיה של השמאל, כמו באוכלוסיה הכללית. למרות ששמאלנים מסוגלים לבקר את הלאומיות כשהיא נמצאת בחצר האחורית שלהם, הם מאבדים את היכולת הזו אל מול תנועות זרות, אקזוטיות.

כל זה משקף את חולשת השמאל. כאשר הוא נאלץ להתמודד עם מלחמה אכזרית וטבח באוכלוסיה באיזורים רחוקים של העולם, הוא מחפש נציג, כתגובה לחוסר יכולת ההשפעה שלו. התמיכה בצד החלש, באנדרדוג~— ב„התנגדות”~— מהווה תחליף להשפעה אמיתית.

ובכל זאת, כאשר אנו נאלצים להתמודד עם מלחמות רחוקות, עלינו להכיר בכך שיש מעט מאוד שאנו יכולים לעשות על מנת להפסיק מלחמה זו או אחרת. לחרם על מוצריה של אחת האומות המעורבות בעימות (כמו בקריאות החוזרות ונשנות להחרים מוצרים ישראלים) יש השפעה זניחה, למרות ש„לעשות משהו” עשוי לגרום לרגשות חיוביים אצל המחרימים. מאבק מעמדי, בזירה של המלחמה ובתוך האומות המתעמתות הוא האסטרטגיה היחידה שאנו יכולים לתמוך בה אם אנו רוצים בעולם ללא מלחמות~— בין אם אלה מלחמות שחרור לאומי, או מכל סוג אחר. מאבק מתוך עמדה מעמדית~— קידום האינטרסים החומריים של מעמד העובדים, במקום להלחם בשדה הקרב של הלאומיות~— הוא זה שיכול לשחרר אותנו מכבלי הלאומיות. לכל הגורמים הלאומיים יש אינטרס משותף למנוע את קיומן של תנועות עובדים עצמאיות, ולארגוני „שחרור לאומי” יש היסטוריה משותפת של דיכוי פעילות עובדים עצמאית~— ה־\L{IRA}, למשל, פעל על מנת לשמור על אחדות בין-מעמדית למען הרפובליקניות האירית על־ידי שבירת שביתות במהלך המאבקים המעמדיים של שנות העשרים של המאה ה־20. ולדוגמה יותר עדכנית, רק לפני זמן קצר, החמאס שבר שביתות של מורים ושל עובדים ממשלתיים. יש להתנגד ללאומיות מכיוון שהיא כובלת את העובדים מאחוריה, ויש לתמוך במאבקים מעמדיים כי הם מציגים את האפשרות לחתוך את הכבל הזה, וסכנת החיתוך מטילה אימה בלב התנועות הלאומיות.

העקרון של לדגול בקו המעמדי, במקום בקו הלאומי, צריך להשליך גם על הפוליטיקה שלנו בארצות בהן אנו חיים. הלאומיות היא כוח חזק, ויש לו השפעה חזקה על מעמד העובדים ברחבי העולם. בבריטניה, שם הזהות והקהילתיות משווקות ומגוייסות תדיר בשיח הרשמי, הצורך להיות שייך לעם, קהילה או קבוצה תרבותית ממלא תפקיד חשוב, ונותן לאנשים מנושלים וחסרי-כוח משהו חשוב להשתייך לו, משהו מעל ומעבר לחדגוניות העגומה של חיי היום-יום. הלאומיות נארזת ונמכרת כמו כל סחורה, היא מהווה מופע של השתתפות בחברה שמוגדרת על־ידי ההפרדה בין הצרכים והתאוות שלנו לבין המניעים לפעילות היום-יומית שלנו. הרעיון של להיות חלק מקהילה, להשתייך למורשת אותה ניתן לתבוע ולהיות מחוברים למשהו שניתן להתגאות בו, ושנמצא מעל ומעבר למציאות המיידית, הוא רעיון מפתה מאד.

לפיכך, הלאומיות יכולה להקיף ולעוות מאבקים מעמדיים: מאבקים חומריים יכולים להפוך למאבקים להגנה על האינטרס הלאומי, למאבקים להכרה בלאום דרך שינוי שיטת המשטר או למאבקים נגד פלגים אחרים של מעמד העובדים המוגדרים על בסיס לאומי, גזעי, או כיתתי. ישנן דוגמאות היסטוריות רבות לשביתות גזעניות נגד עובדים שחורים, נגד מהגרים או מכל סיבה ריאקציונרית אחרת, בין אם אלה עובדי המספנה ששבתו על מנת להגן על אנוך פוול\footnote{\L{Enoch Powell}, פוליטיקאי שמרן באנגליה ובצפון אירלנד. עובדי הנמל המדוברים שבתו במחאה על פיטורו ממשרתו בממשלת הצללים הבריטית, לאחר נאום „נהרות הדם” שלו, בו הביע חשש מפני הצפת אנגליה במהגרים שחורי-עור, במטרה להתנגד לחוק למען שוויון זכויות דיור למהגרים~— המתרגם.}, או השביתה של עובדי מועצת העובדים נאמני אולסטר נגד חלוקת-הכוח בצפון אירלנד\footnote{\L{Ulster Worker's Council strike}, שביתה כללית שהונהגה על־ידי תומכי הפלג הפרוטסטנטי באירלנד נגד הסכם חלוקת הכוח שם, מכיוון שהם חששו שהוא יוביל לאיחוד אירלנד. השביתה ערכה שבועיים, והצליחה להפיל את ההסכם, מה שגרם להחזרת השליטה על צפון אירלנד לבריטניה~— המתרגם.}.

גם מאבקים יום-יומיים יכולים למצוא עצמם צבועים בלאומיות, על־ידי שיח מלא במיתוסים לאומיים ודרך לאומיותם של איגודי העובדים. הופעתם של דגלי לאום בהפגנות, משמרות שביתה ועצרות ברחבי העולם אינה תופעה נדירה.

למרות זאת, התודעה מתפתחת במהלך המאבק. התודעה המעמדית אינה מתפשטת בחברה כתוצאה מהמרתה אינטלקטואלית או דתית של האוכלוסיה לעמדות אנרכיסטיות~— היא אינה באה לידי קיום כתוצאה של ניצחון ב„מלחמת הרעיונות” בזירת הפולמוס הדמוקרטי. התעמולה היא שימושית וחשובה, אך מטרתה היא לגבש מיעוטים פוליטיים שיכולים להשתתף במאבקים, לזכות בהערכה עבור רעיונות אנרכיסטים, וליישם אותם בפועל. תודעה מהפכנית היא תוצר של מאבק עממי, ומאבק מעמדי הוא מעצם מהותו של הקפיטליזם. המאבק העממי הוא זה דרכו התודעה מתפתחת. יש להבהיר, כי תחת הקפיטליזם, נדירים המאבקים ה„טהורים”. המאבק בו מגינים על האינטרסים החומריים של מעמד העובדים, הקשור לדרישות חומריות~— שכר גבוה יותר, פחות שעות עבודה, הגישה לשירותים, ובסופו של דבר, נגד העבודה ונגד הקפיטליזם בכלל~— דרך מאבק זה ניתן לחתוך את כבלי הלאומיות, על־ידי חשיפת חוסר התאימות היסודי בין הצרכים שלנו לבין הצורך של הקפיטליזם להשאר רווחי. האינטרסים הזרים של המעמדות נעשים ברורים במאבקים כאלה, והיכולת להסיק את המסקנה שיש להרוס את המערכת הקפיטליסטית עצמה יכולה לעבור כאש בשדה קוצים, כפי שכבר קרה בעבר.

לקבוצות וארגונים בינלאומנים יש תפקיד חשוב לשחק בהתססה נגד הלאומיות, ובפעילות נגד נטיות לאומיות כאשר הן מתפתחות מרימות את ראשן במהלך מאבקים. עלינו לעמוד איתן כנגד המיליטריזם, הלאומיות והמלחמה, ולהתסיס על בסיס מעשי בהתאם. עלינו לפעול נגד הלאומיות בתוך מעמד העובדים, ולהציע את הסולידריות סביב אינטרסים מעמדיים כדרך המעשית, שבה עובדים יכולים להגן על האינטרסים שלהם. אל מול השמאל, וההצעות שלו, שמסתכמות בארגון מחדש של העולם הקפיטליסטי המורכב ממדינות־לאום, אנו פועלים בנחרצות למען עולם ללא גבולות, ללא אומות וללא מדינות, למען עולם המבוסס על גישה חופשית לתוצרי העשיה האנושית, לטובת סיפוק הצרכים והמאוויים האנושיים. עולם עם שיתוף פעולה וללא מדינות, בו אנשים יכולים לממש את הפוטנציאל המלא שלהם כבריות יצירתיות. במאבק אל עבר המטרה הסופית הזו, אנו מתעקשים על עמדתנו: לעובדים אין מולדת, מעמד העובדים צריך להתאחד על פני כל החלוקות, והסולידריות של כל העובדים היא העקרון בו תלוי כל נצחון עתידי.

לסיום, אנו מציגים כמה הצעות לדרכי פעולה בהן אנרכיסטים יכולים לנקוט כאשר הם מתמודדים עם הלאומיות בארצות בהן הם פועלים, וכאשר הם נתקלים בלאומיות כשהם עוסקים בפעילויות כנגד מלחמות.

ראשית כל, אנרכיסטים מעמדיים צריכים להתארגן במקום העבודה היכן שאפשר, ולקחת חלק בתמיכה בשביתות ובפעולות אחרות התורמות לפיתוח התודעה המעמדית. על אנרכיסטים לפתח קשרים עם פעילים ליברטריאנים אחרים, ובמקומות העבודה עליהם לטעון לטובת טקטיקות ליברטריאניות כגון אסיפות המוניות ופעולה ישירה. בעודם שומרים על פרספקטיבה מעמדית במקום העבודה, עליהם לטעון כנגד חלוקת מעמד העובדים על פי קווים גזעיים או לאומיים, ועליהם לתמוך בסולידריות על פני כל הגבולות, סולידריות שנוטה להתפתח מאליה כאשר עובדים מרקעים שונים נפגשים במהלך מאבק.

בדומה, על אנרכיסטים לטעון כנגד המיתוסים הלאומיים שמפריעים לסולידריות מעמדית מעשית: עליהם להתנגד לשקרים לגבי מהגרים, כאילו הם גונבים עבודות ובתים, ולהציג למולם את המציאות, בה הסיבה לבעיות היום-יומיות שלנו היא העובדה, שהמערכת הקפיטליסטית אינה פועלת על מנת לענות על צרכינו, ומעולם לא נועדה לעשות כן.

שנית כל, אנרכיסטים תמיד לקחו חלק בפעילות נגד המיליטריזם ונגד מלחמות. זה נכון גם כיום, וניתן למצוא אנרכיסטים ברחוב, במסגרת הפגנות כנגד אותן המלחמות, שהן הפועל היוצא של האימפריאליזם. כאשר אנו מוצאים עצמנו מול הטוענים למען שחרור לאומי ולמען תגובה לאומית למלחמה, עלינו להכיר בגועל הלגיטימי שהם חשים כלפי זוועות המלחמה, אך להציג חלופה בינלאומנית, מעמדית, לניתוח הלאומי.

אין אלה מטלות פשוטות, אך הן חיוניות, וחייבות להיות במרכז הפעילות האנרכיסטית כיום.



\section*{נספחים: גילויי דעת לגבי המלחמה בעזה}

\subsection*{המדינה אינה פתרון בעזה (ה־20 לינואר 2009)}

דבר אחד ברור לגמרי לגבי המצב הנוכחי בעזה: ישראל מבצעת מעשי זוועה שחייבים להפסיק מיד. מאות ההרוגים ואלפי הפצועים מבהירים, כי המטרת המבצע הצבאי הזה, שתוכנן עוד מאז החתימה על חוזה הפסקת האש ביוני, היא לשבור את חמאס. התקיפה באה לאחר מצור מצמית לכל אורך „הפסקת האש” לכאורה, מצור שחיסל את אמצעי המחיה של העזתים, הרס את התשתית האזרחית שלהם, ויצר אסון הומניטרי, שמי שלו קמצוץ של אנושיות ירצה להפסיק.

אבל זה לא כל מה שיש לומר על המצב. בשני צידי העימות, הרעיון שלהתנגד לישראל טומן בחובו תמיכה בחמאס ובתנועת „ההתנגדות” שלו הוא נפוץ בצורה מטרידה. אנחנו דוחים טענה זו מכל-וכל. חמאס, בדיוק כמו כל אוסף אחר של שליטים, וכמו כל הפלגים הפלסטינים הגדולים האחרים, מוכן ואף שמח להקריב פלסטינים מן השורה על מנת להגביר את כוחו. לא מדובר באיזו סוגיה תיאורטית עלומה~— לפני זמן לא רב, הייתה תקופה לא מבוטלת בה רוב המתים בעזה היו קורבנות הקרבות בין חמאס לבין פת״ח. ה„ברירות” שניתנו לפלסטינים מן השורה הן בין בריונים איסלאמיסטים (חמאס, הג׳יהאד האיסלאמי) לבין בריונים לאומנים (פת״ח, גדודי חללי אל-אקצא). הקבוצות האלה הראו שהן מוכנות לתקוף את נסיונותיהם של מעמד העובדים לשפר את חייהם, כשהן תפסו משרדי איגודים, חטפו פעילים מובילים של איגודים מקצועיים ושברו שביתות. דוגמה בולטת לכך היא ההתקפה של גדודי חללי אל-אקצא על „רדיו העובדים של פלסטין”, בגלל שהתחנה „ליבתה סכסוכים פנימיים.” נראה ברור כי „פלסטין חופשיה” תחת שליטתם של מי מן הקבוצות האלה לא תהיה כזו בשום צורה.

כאנרכיסטים, אנו בינלאומנים, ומתנגדים לרעיון שלשליטים ולנשלטים בתוך אותה האומה יש אינטרסים משותפים. לפיכך, אנרכיסטים דוחים את הלאומיות הפלסטינית באותה המידה שבה אנו דוחים את הלאומיות הישראלית (הציונות). מוצא אתני לא מקנה „זכויות” לאדמה, שדורשות את המדינה על מנת לאכוף אותם. לאנשים, לעומת זאת, יש הזכות שיספקו את צרכיהם, וצריכים להיות מסוגלים לחיות היכן שירצו, באופן חופשי.

מכאן, בניגוד לחלוקות ולברירות המזויפות שמכוננת הלאומיות, אנו תומכים באופן מלא בתושבים מן השורה בעזה ובישראל כנגד מלחמות מדיניות~— לא בגלל הלאומיות, המוצא האתני או הדת שלהם, אלא פשוט בגלל שהם בני אדם אמיתיים שחיים, מרגישים, חושבים, סובלים ונאבקים. ותמיכה זו חייבת לטמון בחובה עויינות מוחלטת לכל אלה שרוצים לדכא ולנצל אותם~— מדינת ישראל והממשלות והתאגידים המערביים שמחמשים אותה, אבל גם כל פלג קפיטליסטי אחר שרוצה להשתמש בפלסטינים רגילים ממעמד העובדים ככלי משחק במאבקי הכוחות שלהם. הפתרון האמיתי היחיד הוא אחד שהוא קולקטיבי, מבוסס על העובדה שכמעמד, באופן גלובלי, אין לנו דבר פרט ליכולת שלנו לעבוד עבור אחרים, ואנחנו יכולים רק להרוויח מחיסול המערכת הזו~— קפיטליזם~— והמדינות והמלחמות להן היא זקוקה.

העובדה שזה נראה כמו פתרון „קשה” לא הופך אותו לפחות נכון. כל „פתרון” שהמשמעות שלו היא מעגלים חוזרים ונשנים של עימות~— שהרי זה מה שהלאומיות מייצגת~— אינו פתרון כלל. ואם כך הדבר, העובדה שהוא „פשוט יותר” אינה רלוונטית. ישנם מגזרים בחברה הפלסטינית שאינם נשלטים על־ידי אלה המתיימרים להגיע אל השלטון~— הפגנות המאורגנות על־ידי ועדות כפריות בגדה המערבית, למשל. להם מגיעה תמיכתנו, כמו גם לאלה בישראל שמסרבים להלחם, ושמתנגדים למלחמה. אבל לא לקבוצות הקוראות לפלסטינים להטבח בשמן על־ידי אחד הצבאות המתקדמים ביותר בעולם, ואשר מוכנות לכוון אל אזרחים בצידו השני של הגבול.

לא משנה מי ימות, החמאס ומדינת ישראל ינצחו

סולידריות עם קורבנות המלחמה (ה־25 לינואר 2009)

הזוועה בעזה

בשוך הקרבות, היקף מעשי הזוועה שביצעה מדינת ישראל כנגד האוכלוסיה ברצועת עזה נעשה ברור. אלפי מתים, שנהרגו בהפצצה הפראית, על אחד מהאיזורים המאוכלסים ביותר על פני כדור הארץ. ישראל השתמשה בתחמושת הזרחן הלבן האסורה באיזורים אזרחיים, ירתה פגזים על משלחות סיוע, על בתי ספר, על מקומות מחסה ועל מסגדים מלאים באנשים. היא הרסה מאגרי סיוע בפגזי זרחן לבן. יותר מ־90,000 אנשים נותרו ללא קורת גג. הכלכלה והתשתיות בעזה, שכבר ניזוקו מן המצור, נהרסו כליל. משנחתמה הפסקת האש, המצור המתמשך יביא להמשך המלחמה נגד האוכלוסיה האזרחית באמצעים אחרים.

פתרון שתי מדינות?

בזמן פול מטר הפצצות, כל מפלגה וכל קבוצה הציגה את החזון שלה ל„פתרון” הבעיה, והחזון שלה לעתיד הפלסטינים. אבל כדי להבין מה אנחנו יכולים לעשות, הצעד הראשון הוא לעבין מה איננו יכולים לעשות. עלינו להבהיר מהן הדרכים שבהן אנו יכולים למנוע את הישנותן של זוועות כאלה.

„פתרון שתי מדינות” על בסיס גבולות 1949 או 1967 לא יוכל להתממש אלא דרך שינוי רחב-היקף במערך הכוחות העולמי. זה יוביל בהכרח לעימותים נוספים במקומות אחרים. שתי מדינות עם הגבולות כפי שהן כעת יובילו לפלסטין שנמצאת תחת שליטה ישראלית, בדיוק באותה המידה שהשטחים נשלטים על ידה כיום. גם אם „פתרון המדינה האחת” יהפוך למציאות, מעמד העובדים הפלסטיני יישאר תת-מעמד של פועלים זולים. זה יהיה כמו הסוף של האפרטהייד בדרום אפריקה. הצבע של מי שבשלטון השתנה, אבל רוב האוכלוסיה נשארה עם אותו העוני ואותו חוסר-התקווה מהם סבלו לפני-כן.

בנוסף, נכון הוא שאיננו יכולים לקרוא למדינה „שלנו” לרסן את ישראל. ראשית, המדינה לא תתן לנו דבר אלא אם מעמד העובדים~— רובנו, ששורדים רק בזכות היכולת שלנו לעבוד עבור אחרים~— יהיה בטוח מספיק בעצמו על מנת לכפות על המדינה להכנע לדרישותנו דרך פעולה קולקטיבית. שנית, הרי זה טירוף לצפות מבריטניה לכפות התנהגות „מתורבתת” על בת-ברית כגון ישראל. בריטניה לקחה חלק בכיבוש בעירק, שגרם למותם של מליון, שלושים ושלושה אלף אנשים. המדינה היחידה שיש לה יכולת כלשהי לרסן את ישראל היא ארה״ב. ארה״ב תעשה זאת רק כאשר מעשיה של ישראל יאיימו על האינטרס הלאומי שלה. עלבון מוסרי לא יביס את הצורך לשלוט באיזור.

\subsection*{סולידריות עם מאבקי מעמד העובדים}

עלינו לעמוד איתן בסולידריות עם קורבנות מלחמת המדינה. האוכלוסיה המבוהלת של עזה לא שעתה לקריאות חמאס להתנגד תוך הפיכה ל„קדושים מעונים”, או לבצע פיגועי התאבדות. הם ברחו בהמוניהם. הם לא הראו שום רצון לבצע „התנגדות” בשמם של אדוניהם, כאשר משמעותה היא מוות בטוח. בעוד הפלסטינים ברחו מן המתקפה, אלה שסרבו לשרת את מכונת השמד ערכו הפגנות בישראל. הסירובים האלה לשעות לקריאה לקרב של המדינה או של המפלגה השלטת ראויים לתמיכתנו ולסולידריות שלנו.

איננו יכולים לתמוך בחמאס, או בכל פלג אחר בעזה או בגדה המערבית נגד ישראל, גם „באופן ביקורתי”. העבר של חמאס, בדיכוי נסיונותיהם של עובדים לשפר את תנאי חייהם, אינו סוד. הם ליוו מורים שובתים חזרה לעבודתם באיומי רובים, וסגרו מרכזים רפואיים כשהסגל הרפואי ניסה לשבות. גם חמאס וגם פתח ביצעו נסיונות חטיפה והתנקשות כלפי אותם פעילי איגודים מקצועיים. חמאס מוציאים להורג את אלו שנדחפים מכורח הנסיבות לעבוד בשירותי מין, ורודפים אחר הומואים ולסביות. אין להם יותר או פחות להציע לפלסטינים מן השורה מאשר שיש ליריביהם החילונים, כמו גדודי חללי אל-אקצא, שתקפו את רדיו עובדי פלסטין באשמת „ליבוי עימותים פנימיים.” בינלאומנות אמיתית משמעה ההכרה שלשולטים ולנשלטים בתוך „אומה” אין שום דבר במשותף. במקרה זה, המשמעות היא לתמוך במאמציהם של פלסטינים מן השורה לשפר את תנאי מחייתם. אנו תומכים בהם, בין אם כנגד ישראל, כמו במאבקים שמאורגנים על־ידי ועדות כפריות בגדה המערבית, או כנגד ארגוני „ההתנגדות” שמפקחים על האוכלוסיה.

הסולידריות שלנו צריכה להיות עם קורבנות המלחמה. אלה ברובם המוחץ פלסטינים, אך גם עובדים, יהודים, ערבים ואחרים, שנהרגו על־ידי פצצות מרגמה ורקטות בישראל. אסור שזה ינבע מהגזע, הלאום או הדת שלהם; זה חייב להיות בגלל שהם בני אדם שחיים, חושבים, מרגישים ונאבקים. ואנו חייבים לעמוד איתן כנגד כל אלה שרוצים להקריבם למטרותיהם שלהם. בסופו של דבר, הפתרון היחיד לסכסוכים ומלחמות עולמיים אינסופיים הוא שאנחנו, אנשי מעמד העובדים, הרוב המנושל, שחייב למכור את הזמן והאנרגיה שלו לאלה שהם בעלי החברה ושולטים בה, נאבק למען האינטרסים שלנו באופן קולקטיבי, נגד הניצול שלהם, ונגד חלוקות כגון מגדר וגזע. המשמעות היא מאבק כנגד המערכת הקפיטליסטית שיוצרת מלחמות במהותה, ושחייבת לנצל אותנו על מנת לשרוד. מכאן נוכל להתחיל לקחת לעצמנו שליטה על חיינו שלנו, ולשים קץ לעולם של מדינות לוחמות, ומדינות בהתהוות, שיצר זוועות כגון אלה שהיו בעזה.
